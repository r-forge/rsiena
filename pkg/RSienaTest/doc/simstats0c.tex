\documentclass[12pt,a4paper]{article}
\usepackage[pdftex,dvipsnames]{color}
\usepackage{graphicx,times}
\usepackage{amsmath}
\usepackage{amsfonts}
\usepackage{amssymb}
\usepackage{algorithm}
\usepackage{algorithmic}
\usepackage[pdfstartview={}]{hyperref}
\textheight=9.5in
\topmargin=-0.5in
\newcommand{\eps}{\varepsilon}
\newcommand{\abso}[1]{\;\mid#1\mid\;}
\renewcommand{\=}{\,=\,}
\newcommand{\+}{\,+\,}
% ----------------------------------------------------------------
\newcommand{\remark}[1]{\par\noindent{\color[named]{ProcessBlue}#1}\par}
\newcommand{\mcc}[2]{\multicolumn{#1}{c}{#2}}
\newcommand{\mcp}[2]{\multicolumn{#1}{c|}{#2}}
\newcommand{\nm}[1]{\textsf{\small #1}}
\newcommand{\nnm}[1]{\textsf{\small\textit{#1}}}
\newcommand{\nmm}[1]{\nnm{#1}}
\newcommand{\R}{{\sf R }}
\newcommand{\sfn}[1]{\textbf{\texttt{#1}}}
\newcommand{\Rn}{{\sf R}}
\newcommand{\rs}{{\sf RSiena}}
\newcommand{\RS}{{\sf RSiena }}
\newcommand{\SI}{{\sf Siena3 }}
\newcommand{\Sn}{{\sf Siena3}}
% no labels in list of references:
\makeatletter
\renewcommand\@biblabel{}
\makeatother

\hyphenation{Snij-ders Duijn DataSpecification dataspecification dependentvariable ModelSpecification}

% centered section headings with a period after the number;
% sans serif fonts for section and subsection headings
\renewcommand{\thesection}{\arabic{section}.}
\renewcommand{\thesubsection}{\thesection\arabic{subsection}}
\makeatletter
 \renewcommand{\section}{\@startsection{section}{1}
                {0pt}{\baselineskip}{0.5\baselineskip}
                {\centering\sffamily} }
 \renewcommand{\subsection}{\@startsection{subsection}{2}
                {0pt}{0.7\baselineskip}{0.3\baselineskip}
                {\sffamily} }
\makeatother

\newcommand{\ts}[1]{\par{\color[named]{Red}TS: #1}\par}

\renewcommand{\baselinestretch}{1.0} %% For line spacing.
\setlength{\parindent}{0pt}
\setlength{\parskip}{1ex plus1ex}
\raggedright
\begin{document}

\title{Simstats0c/FRAN}
\author{Ruth Ripley}
\date{}
\maketitle

\centerline{\emph{\today}}
\bigskip
\section{Introduction}

The \R function \nnm{simstats0c} (so named during development when I had other
versions which did not call C!) is the function which controls the
simulations. It is used as an argument to the function \nnm{sienaModelCreate}
and stored as the element named \nm{FRAN} of the model object. It has three (or
3+) modes: initial, to set up the C++ data structure, terminal (to tidy up) and
ordinary (to do one complete simulation). (There is an extra mode for using
multiple processors, which does some of the work of an initial call.) It uses
C++ functions for most of its work.

\section{Outstanding work}
\begin{enumerate}
\item When setting up the effects for behaviour variables, it is
  assumed that the one-mode network is non-symmetric. Should not!
\item Allow data values indicating structurally fixed values to be a parameter.
\item Validate things
\item Sort out structural value processing
\end{enumerate}

\section{Model specification}

\subsection{The data as input to \nnm{simstats0c}}

A \nnm{siena} (or \nnm{sienaGroup}) object and a \nnm{sienaEffects} (or
\nnm{sienaGroupEffects}) object.


\subsection{The model as it reaches C++}

\begin{description}
\item [Virtual data] A set of dependent variables, actor sets representing the
  nodes for the variables, and covariates of various types. Never actually
  exist, but the \R routines must check that each data object is consistent with
  the virtual data.
\item[Data] Each data object consists of several waves for each of the dependent
  variables, each of which uses some subset of each corresponding actor set,
  together with covariate objects of appropriate sizes.
\item[Edgelists] Networks and dyadic covariates are passed to C++ as valued
  edgelists.
\item [Networks] Networks have three edge lists: data, missing, structural
  values. Missing and structural values have been replaced in the data.
\item[Behaviour variables] Passed in as a pair of matrices, data and missing
  entries (as 1's). Missing values have been replaced in the data matrix, as
  have structural values(?)
\item [Covariates] Missing values have been replaced. Node set name(s) are
  attached to the object as attributes.
\item[Joiners and leavers] This is for actors joining or leaving the network.
  Data is reformatted to be, for each actor set, a list
  of two items. The first is a data frame with columns event type, period, actor
  and time. The second is a matrix of booleans, indicating for each actor
  whether active at the start of each period.
\item[Uponly, downonly] Uponly and downonly flags are attached to all dependent
  variables as attributes, one value for each pair of observations.
  They indicate that the dependent variable must not decrease, or must not
  increase, respectively.
\item[symmetric] flag for each one-Mode network dependent variable. Only true if
  true for all observations of the virtual network.
\item[maxdegree] one integer per dependent network, same value used for each
  instance of the virtual network.  For valued graphs (not yet!), maxdegree will
  refer to the number of non-null outgoing ties per actor.
\item[derivs required] flag to suppress calculating the scores
\item[from finite differences] flag to indicate that this run is a `repeat'
  which needs to reuse the random numbers of a previous run. Implemented by
  obtaining from \R and storing the random number objects (\sfn{.Random.seed})
  at the start of the simulation for each period in each non finite difference
  run and restoring the random number object into \R in each finite difference
  run. The values are stored in the \nm{FRANstore} function in between!
\item [Effects] A data frame containing identification information for
  each effect. In the setup phase the effects objects are created and pointers
  to objective function effects are stored on this data frame. In simulation
  calls, the pointers are used to access these effects to update the
  parameters. Rate effects are accessed differently.

  A single effects object contains all the effects for a multi-group project:
  some of the basic rate effects need to be selected out when reading the data
  frame for particular data object.

  The data frame also has columns which could be used for pairs of R functions
  (for user-defined effects) (one each for effect and statistic calculation, to
  be passed back to R, a simple way to implement user-defined effects where
  speed is not critical). This interface for user-defined effects is not
  currently implemented.
\item[balmean] A calculation for each network, at virtual data level .
\item[similaritymean] A calculation for each behavior variable or covariate, at
  virtual data level.
\item[range] For each behavior variable and covariate, at virtual data level
\item[mean] For each dyadic covariate, at virtual data level.
\end{description}
\section{Initial call to simstats}
\begin{itemize}
\item Convert data and pass to C++
\item Select requested effects and pass to C++
\item Pass Model options to C++
\item Check consistency of networks with each other and with covariates. Need at
  least one dependent variable. (a check that data has been interpreted
  correctly!) (Not really done yet!)
\end{itemize}
\section{The Loop Processing}
\begin{algorithmic}
\STATE Update the parameters for the effects
\FOR {each group}
\STATE Create new EpochSimulation object
\FOR {each period (omitting the final one)}
\IF {from Finite Differences}
\STATE restore the random number seed
\ELSE
\STATE store the random number seed
\ENDIF
\REPEAT
\STATE Get \nnm{tau} (section \ref{sec:tau})\\
\IF{the time is \textcolor{red}{greater} than the time to the next
 joiner or leaver in any network}
\STATE change the composition of the networks (section \ref{sec:cc})\\
set the time so far to the time of the composition change event\\
\texttt{next iteration}
\ENDIF
\STATE Choose dependent variable to alter with probabilities
proportional to the total lambda in each dependent variable.\\
\STATE For symmetric networks totlambda is defined diferently\\
\STATE Choose actor proportional to lambda\\
(For symmetric networks use lambda and sum of lambdas as usual, even if
totlambda was calculated differently)\\
\IF{\nnm{deriv}}
\STATE Accumulate rate scores Section \ref{sec:scores}.
\ENDIF
Choose change \ref{sec:change}\\
Update state\\
If required, add scores to accumulators
\UNTIL {time or change is reached, or decide it will not be reached}
\STATE Calculate statistics (section \ref{sec:stats})
\ENDFOR
\ENDFOR
\RETURN statistics, scores (if requested), total time (if conditional),
simulated networks (if requested, only 1 at the moment!)
\end{algorithmic}
\subsection{Get \nnm{tau}}
\label{sec:tau}
\begin{algorithmic}
\FOR[calculate totlambda]{each dependent variable}
\FOR[calculate lambda] {each actor}
\IF{inactive $\equiv$ all
links structurally fixed in this period or not yet joined
  the network}
\STATE lambda = 0
\ELSE
\STATE lambda = product of basic rate for the period with other
selected rate effects multiplied by relevant theta value
\ENDIF
\STATE Add to totlambda
\ENDFOR
\STATE Add to sum of totlambda
\ENDFOR
\IF{model type BFORCE, BAGREE, BJOINT}
\STATE Replace totlambda by square of totlambda - sum of squared lambdas
\ENDIF
\STATE Get a random exponential with sum of totlambda as parameter.\\
\end{algorithmic}
\subsection{Choose Change}
\label{sec:change}
\begin{algorithmic}
\STATE Define alters =$\{1\ldots \nnm{dim2} \}$
\IF{not symmetric or model type is A}
\IF{bipartite}
\STATE add \{0\} to  alters
\ENDIF
\FOR[calculate deltas and effects]{$j \in \text{alters}$ }
\IF{out-degree of actor = \nm{maxdegree}}
\STATE change is only permissible if there is a link to this alter, to
change its value
\ELSIF{link from actor to $j$ is structurally determined}
\STATE $\nnm{delta} = \{\}$
\ELSE[changes are permissible]
\IF{behaviour variable}
\STATE $\nnm{delta} = \{+1, -1\}$
\COMMENT{may allow larger changes}
\IF {uponly }
\STATE$\nnm{delta} = \{+1\}$
\IF {equal to maximum}
\STATE $\nnm{delta} = \{\}$
\ENDIF
\ELSIF {downonly}
\STATE $\nnm{delta} = \{ -1\}$
\IF{value = minimum}
\STATE $\nnm{delta} = \{\}$
\ENDIF
\ELSIF {value = maximum}
\STATE  $\nnm{delta} = \{ -1\}$
\ELSIF {value = minimum}
\STATE$\nnm{delta} = \{+1\}$
\ENDIF
\ELSE[not behaviour variable]
\IF{j = 0 or j = actor}
\STATE $\nnm{delta} = 0$
\ELSE
\STATE $\nnm{delta} = \{1-2\nm{y}\}$
\COMMENT{for valued networks, this would alter}
\ENDIF
\ENDIF
\ENDIF
\STATE calculate \nnm{p(j, delta)} effect of each
\nnm{delta} on the objective function (sum of effects multiplied by
corresponding parameters)\\
\ENDFOR
\STATE Choose change proportional to $\exp(\nnm{p(j,delta)})$
\IF{symmetric and model type is A and subtype is 2 (alias M1 in PA paper)
  and alter is not equal to actor}
\STATE Check if alter agrees Section \ref{sec:alterSymA}
\IF{alter does not agree}
\STATE abort the change
\ENDIF
\ENDIF
\ELSE[symmetric type B  (alias M2, D2, C in PA paper) ]
\STATE Choose alter with same probabilities as actor
 (repeat until not equal to actor)
\\ Calculate probabilities of acceptance. Section\ref{sec:alterSymB}
\IF{not accepted change}
\STATE abort the change
\ENDIF
\ENDIF
\IF{\nnm{deriv}}
\STATE Calculate score functions.
Section \ref{sec:scores}.
\ENDIF
\end{algorithmic}
\subsection{Symmetric network models}
There are various types in two groups: Type A or type B. In type A the change
is selected in the normal way, but for subtype 2 the alter gets a chance to
say no. In type B we select the alter using the rate probabilities and then
simply consider that one link.\\
In the PA paper, Type A models have new names:\\
A subtype 1 = forcing = D1,\\
A subtype 2 = unilateral \& reciprocal confirmation = M1.\\
Type B models are called:\\
BAGREE = B subtype 1 = pairwise conjunctive = M2;\\
BFORCE = B subtype 2 = pairwise disjunctive = D2;\\
BJOINT = B subtype 3 = pairwise compensatory = C (same as tie-based).
\par
Section \ref{sec:change} referred to the following additional
steps in the case of a symmetric network.
\subsubsection{Decision probabilities}
Depending on options and outcomes, we may in the sequel
need probabilities $p_{ij}$ or $p$ as defined now.
These probabilities are defined for each relevant link from the
perspective of the appropriate actor, by
\begin{algorithmic}
\STATE Calculate the change contributions for each effect for this link from
$i$ to $j$, from $i$'s perspective.
\STATE Calculate $y_{i,j}$, the unnormalised probability by multiplying by the
parameters and summing.
\IF{link between $i$ and $j$ exists}
\STATE change the sign of the unnormalised probability $y_{i,j}$
\ENDIF
\IF{subtype not 3}
\STATE Calculate $$p_{i,j} = \frac{\exp(y_{i,j})} { (1 + \exp(y_{i,j}))}$$
\ELSE
\STATE Calculate $$p = \frac{\exp(y_{i,j} + y_{j,i})}
{(1 + \exp(y_{i,j} + y_{j,i}))}$$
\ENDIF
\end{algorithmic}
\subsubsection{Model Type A subtype 2}
\label{sec:alterSymA}
\begin{algorithmic}
\IF{the proposal implies the creation of a tie}
\STATE Calculate the probability $p_{alter, actor}$ as defined above.
\IF{a random number is less than $p_{alter, actor}$}
\STATE alter agrees
\ELSE
\STATE alter does not agree
\ENDIF
\ENDIF
\end{algorithmic}
\subsubsection{Model Type B}
\label{sec:alterSymB}
\begin{algorithmic}
\IF{BFORCE or BAGREE}
\STATE calculate $p_{actor, alter}$ and $p_{alter, actor}$
as described above
\ELSIF{BJOINT}
\STATE calculate $p$ as described above
\ENDIF
\IF[either can force existence]{BFORCE}
  \IF{link between $i$ and $j$ exists}
  \STATE accept if product of probabilities is greater than a random number
  \ELSE
  \STATE accept if $p_{actor, alter} + p_{alter, actor} - p_{actor, alter} *
p_{alter,actor}$ is greater than a random number.
  \ENDIF
\ELSIF[both must agree for existence]{BAGREE}
  \IF{link between $i$ and $j$ exists}
  \STATE accept if $p_{actor, alter} + p_{alter, actor} - p_{actor, alter} *
p_{alter,actor}$ is greater than a random number.
  \ELSE
  \STATE accept if product of probabilities is greater than a random number
  \ENDIF
\ELSE [BJOINT, joint decision]
\STATE accept if p is greater than a random number
\ENDIF
\end{algorithmic}
\subsection{Calculation of scores}
\label{sec:scores}
For symmetric networks other than Model Type A, see section
\ref{sec:symScores}. For symmetric networks with Model type A and subtype 2, the
rate scores are the same as in this section, but the non-rate scores are defined
in section \ref{sec:symScores}.

In \Sn, when calculating scores, only steps which are not aborted because of
composition change are included in the scores. (This is a bug in \Sn.)

Let $z_{rk}$ be the observed value of the $k$th
statistic in the $r$th simulation\\
$r_m$ the total rate for this dependent variable at step $m$ of a simulation\\
$r_M$ the total rate for this dependent variable at the generation of the final
time interval in the unconditional case\\
$t_m$ the \nnm{tau} of the $m$th step (maybe a composition change)\\
$s_{ikm}$ be the value of the $k$th statistic for the $i$th actor at
the $m$th step\\
$r_{im}$ be the total rate for actor $i$ at the $m$th step\\
$s_{i,j\delta k}$ be the change in the $k$th statistic corresponding to
choice of $i$ as actor and $j$ as alter with change \nnm{delta}\\
$p_{i,j \delta }$ the probability of selecting the change \nnm{delta} with actor
$i$ and alter $j$\\
$c_{rk}$ the score corresponding to $\theta_k$ in the $r$th
simulation\\
$i$ the actor, $j$ the alter at the $m$th step of the total $M$.
$t_c$ is the time without an event due to composition change
(Does not matter
whether $z_{rk}$ is statistic or deviation from observed, provided the
same in each term!).
\begin{align*}
\intertext{for a basic rate parameter, $(\lambda)$, for a
dependent variable,}
c_{rk} &= \sum_{m=1}^M \left(
  \frac{\delta_{pq}}{\lambda_m} - r_m t_m / \lambda_m\right ) - \delta_{bc}
r_M  \left (1- \sum_{m=1}^M t_m - \sum_ct_c
  \right)/\lambda_m\\
\intertext{for other rate parameter}
c_{rk}&= \sum_{m=1}^M \left [\delta_{pq} s_{ikm}- \sum_{d=1}^n s_{dkm} r_{dm}t_m\right]\\
& \qquad - \delta_{bc}\sum_{d=1}^n s_{dkM}
r_{dM}\left (1- \sum_{m=1}^M t_m  \right)\\
\intertext{for other parameters}
c_{rk}&= \sum_{m=1}^M \left [ s_{i,j\delta k}- \sum_{a,\nnm{delta}}
  s_{i,a\delta k}p_{i,a\delta}\right]\\
\intertext{with derivative matrix}
 D_{jk}&=\frac{1}{R}\sum_{r=1}^R z_{rj}c_{rk}
 - \frac{1}{R^2}\sum_r z_{rj} \sum_r c_{rk}\\
\end{align*}
where $\delta_{pq}$ is 1 if the rate parameter is selected and 0 otherwise (0
for all dependent variables if there is composition change), $\delta_{bc}$ is 1
if the simulation is unconditional and 0 otherwise, and $R$ is the number of
simulations performed (considering each period as a separate simulation).

Addendum: for maximum likelihood we need derivatives of scores too. Using
similar notation, with $v_{rk_1k_2}$ the derivative of the score corresponding
to$(\theta_{k_1}, \theta_{k_2})$  in the $r$-th chain,  this is
\begin{align*}
\intertext{for a basic rate parameter, $(\lambda)$, for a
dependent variable, not the total here, just the per actor rate}
v_{rkk} &= \sum_{m=1}^M  \textrm{nbr of non-structurally determined links in the
  chain}/ \lambda_m^2\\
\intertext{for other parameters}
v_{rk_1k_2}&= \sum_{m=1}^M \left [  \left( \sum_{a, \nnm{delta}_a}
  s_{i,a\delta k_a} p_{i,a\delta} \right) \left (
\sum_{ b, \nnm{delta}_b} s_{i,b\delta k_b} p_{i,a\delta} \right) -
\sum_{a, \nnm{delta}}
  s_{i,a\delta k_1}p_{i,a\delta}  s_{i,a\delta k_2}
\right ]\\
\intertext{with derivative matrix}
 D_{jk}&=\frac{1}{R}\sum_{r=1}^R v_{rjk}
\end{align*}

Note that non-basic rate effects are not allowed in the maximum likelihood
case and basic rate effects are uncorrelated with all other effects.

\subsection{Symmetric network scores}
\label{sec:symScores}
\subsubsection{Model type A, subtype 2 (AAGREE)}
Here we have added a term to the log likelihood, accept or reject with
probability $p_{ji}$.
The scores for the non rate effects will be

$$c_{rk}= \sum_{m=1}^M \left [ s_{i,j\delta k}- \left(\sum_{a,\nnm{delta}}
  s_{i,a\delta k}p_{i,a\delta}\right) +
p_{ji}^{1-t} (1-p_{ji})^t (-1)^{1-t}s_{j,i\delta
  k}\right]$$

Where $t$ is 1 if agree to change, 0 otherwise.
\subsubsection{Model type B}
Here we do not do the normal choice. My calculations suggest that for the basic
rates it should be twice the normal scores using total rate and basic rate as
usual. (Within variable).

For the other rate effects, we have

$$
    \text{log likelihood term } = \delta_{pq} \left [\log(f) + \log r_{im}
    - \log(\sum r_{dm})\right ] -ft
$$
where
$$f=(\sum r_{dm})^2 - \sum r_{dm}^2$$

$$\frac{\partial f}{\partial \alpha_k} =
2 \sum r_{dm} \sum r_{dm} s_{dk} - 2 \sum r_{dm}^2 s_{dk}
$$

so the scores for other rate effects will be:

$$
c_{rk} = \delta_{pq}\left [ \dfrac{1}{f} \frac{\partial f}{\partial \alpha_k}  + s_{ik}
- \dfrac{1}{\sum r_{dm}}\sum
r_{dm} s_{dk} \right ]-
t\frac{\partial f}{\partial \alpha_k}
$$

But I have not coded this yet!

The scores for the non rate effects will be:

For subtype 1 (BFORCE):

4 different cases: whether there is a link from $i$ to $j$, and whether the
change is accepted or not.

$$
c_{rk}= \sum_{m=1}^M \begin{cases}
 (1-p_{ij})s_{i,j\delta k} + (1-p_{ji}) s_{j,i\delta k}
& \exists \text{ link }  i \rightarrow j, \text{ accept}\\
\dfrac{ -p_{ji} p_{ij}}{(1 - p_{ij} p_{ji})}
\left[(1-p_{ij}) s_{i, j\delta k} +  (1-p_{ji})
s_{j, i\delta k}\right ] & \exists \text{ link }
i \rightarrow j, \text{ reject}\\
\dfrac{(1 - p_{ij}) (1-p_{ji} )\left [s_{i, j\delta k} p_{ij}  +
 s_{j, i\delta k} p_{ji} \right] }
{p_{ij} + p_{ji} - p_{ij}p_{ji}}& \nexists \text{ link } i \rightarrow j,
\text{ accept}\\
-s_{i, j\delta k} p_{ij} -
s_{j, i\delta k} p_{ji}
 &\nexists \text{ link } i \rightarrow j,
\text{ reject}
\end{cases}
$$

For subtype 2 (BAGREE), the reverse:
$$c_{rk}= \sum_{m=1}^M \begin{cases}
 (1-p_{ij})s_{i,j\delta k} + (1-p_{ji}) s_{j,i\delta k}
& \nexists \text{ link }  i \rightarrow j, \text{ accept}\\
\dfrac{ -p_{ji} p_{ij}}{(1 - p_{ij} p_{ji})}
\left[(1-p_{ij}) s_{i, j\delta k} +  (1-p_{ji})
s_{j, i\delta k}\right ] & \nexists \text{ link }
i \rightarrow j, \text{ reject}\\
\dfrac{(1 - p_{ij}) (1-p_{ji} )\left [s_{i, j\delta k} p_{ij}  +
 s_{j, i\delta k} p_{ji} \right] }
{p_{ij} + p_{ji} - p_{ij}p_{ji}}& \exists \text{ link } i \rightarrow j,
\text{ accept}\\
-s_{i, j\delta k} p_{ij} -
s_{j, i\delta k} p_{ji}
 &\exists \text{ link } i \rightarrow j,
\text{ reject}
\end{cases}
$$

For subtype 3 (BJOINT):

$$
c_{rk}= \sum_{m=1}^M  \begin{cases}
(1-p)(s_{i,j\delta k} + s_{j,i\delta k} ) & \text{accept}\\
-p (s_{i,j\delta k} + s_{j,i\delta k}) & \text{ reject}
\end{cases}
$$

\subsection{Calculate statistics}
\label{sec:stats}
\begin{itemize}
\item set leavers and structurally determined values of
  network links back to the ones in the stored data as at
    start of current period (so don't count in change
  etc.)
\item temporarily set the behaviour variables back to the beginning
of the simulation while calculate the network statistics and vice
versa
\item when calculating the statistics for any period, ignore any actor with
  missing information at either end of the period. Except for endowment
  statistics where the end of period values are not relevant.  (In any network,
  any statistic, or just those which are used in the current calculation? Only
  looks at current network at the moment. Not done by missing flags for the
  actor.)
\end{itemize}
\subsection{Change network composition}
\label{sec:cc}
\begin{algorithmic}
\IF{leaver}
\STATE set active flag  to false\\
 remove any links current\\
Reset Structactive flags for all other active
  actors. (only structactive if have a link which is not structurally
  determined to an active actor, and you've just set an actor
  inactive.)\\
\ELSIF{joiner}
\STATE Set active flag to true\\
activate ties from the stored networks into
  the current simulated one.\\
 reset structactive flags
\ENDIF
\STATE update current average mean of the behaviour variables, which excludes
structurally inactive actors. Also update the time so far to the time at which
the composition changes.
\STATE update the scores if necessary
\end{algorithmic}

\section{Prior processing in R}
\begin{description}
\item[starting values] Note that structural values are excluded from the
  distance but not from the total cell count (and missing cell count is done by
  subtraction). Need care to get these sorted!
\item[effects] user requests - validate as appropriate. If there are
  structural zeros will need to check whether some effects should be
  removed. (not done yet!)
\item[missing data] Create a separate edge list, and replace in the original one
  by 0 or any value in a preceding network. Similarly for
    behavior variables: impute by 0 or last previous observation.
\item[structural zeros and ones] Create a separate edge list, and replace by
  (value-10) in the original one. To do: Make the value 10 a parameter,
    with any lower values non-structural and those higher structural.
\item[if conditional]
remove the relevant rate parameter from the working copies of
theta\\
Remove corresponding targets, and associated flags\\
Divide the other basic rate parameters by the one removed\\
Set up array to store the times used in simulations in phase 3 from
which to estimate the parameter conditioned on\\
Extract relevant targets for use as distance.
\item[joiners and leavers] Set the values in the networks in
  accordance with the option requested in the input flag, and the
  pattern, as below.
\begin{itemize}
\item Can only do unconditional estimation with joiners or leavers.
\item Input format is a list of vectors, one for each actor,
containing the start and end of (inclusive) intervals during which the actor was
present.
\item Attributes are attached to the object, (created in \sfn{sienaDataCreate})
  containing:
\begin{description}
\item[action] matrix with row for each actor, column for each observation.
Values are \begin{enumerate}
\item  Imputed or zeroed data is not preceded by real data
\item  Imputed or zeroed data is both preceded and followed by real data
\item  Imputed or zeroed data is not follwed by any real data
\end{enumerate}
\item[event] data frame with details of the events in the format required for
  C++ (see above!) (Actually exclude events at less then time 1e-10 here).
\item[activeStart] Matrix of booleans with row for each actor, column for each
  observation. TRUE if the actor was active at the start of the period. passed
  unchanged to C++.
\end{description}
\item alter the network data, using the
  \sfn{action} attribute.
\begin{algorithmic}
\FOR {each network using this actor set (not behavior variables)}
\FOR {each period}
\FOR {each actor with an \emph{action} value of 1}
\IF {the composition change option is 1 or 2}
\STATE zero the data in the value edgelist\\
remove the entries from the missing data edgelist\\
remove from the data to be used to calculate distances for the report
\ELSE[option is 3]
\STATE add to missing data edgelist
\STATE mark as missing in data to be used to calculate distances for the report
\ENDIF
\ENDFOR
\FOR {each actor with an \emph{action} value of 3}
\IF {the composition change option is 1 or 2}
\STATE carry forward the data in the preceding edgelist\\
remove the entries from the missing data edgelist\\
set to 0, not missing in the data to be used to calculate distances for the report
\ELSE[option is 3]
\STATE add to missing data edgelist
\STATE mark as missing in data to be used to calculate distances for the report
\ENDIF
\ENDFOR
\FOR {each actor with an \emph{action} value of 2}
\IF {the composition change option is 1}
\STATE carry forward the data in the preceding edgelist\\
remove the entries from the missing data edgelist\\
set to 0, not missing in the data to be used to calculate distances for the report
\ELSE[option is  2 or 3]
\STATE add to missing data edgelist
\STATE mark as missing in data to be used to calculate distances for the report
\ENDIF
\ENDFOR
\ENDFOR
\ENDFOR
\end{algorithmic}
\item
Check for empty sets of active actors:
\begin{itemize}
\item for each pair of consecutive periods, count number active at start of
  both.
\item If this is less than 2,
\begin{itemize}
\item count how many active at start
\item go through the events to find if at any stage there is only one or
  none left
\item If so, quit with an error message that one or none is left.
\end{itemize}
\end{itemize}
\end{itemize}

\end{description}


\end{document}
