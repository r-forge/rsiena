\documentclass[12pt,a4paper]{article}
\usepackage[pdftex,dvipsnames]{color}
\usepackage[noend]{algorithmic}
\usepackage{graphicx}
\usepackage{amsmath}
\usepackage{amsfonts}
\usepackage[pdfstartview={}]{hyperref}
\newcommand{\eps}{\varepsilon}
\newcommand{\abso}[1]{\;\mid#1\mid\;}
\renewcommand{\=}{\,=\,}
\newcommand{\+}{\,+\,}
% ----------------------------------------------------------------
\newcommand{\remark}[1]{\par\noindent{\color[named]{ProcessBlue}#1}\par}
\newcommand{\mcc}[2]{\multicolumn{#1}{c}{#2}}
\newcommand{\mcp}[2]{\multicolumn{#1}{c|}{#2}}
\newcommand{\nm}[1]{\textsf{\small #1}}
\newcommand{\nnm}[1]{\textsf{\small\textit{#1}}}
\newcommand{\nmm}[1]{\nnm{#1}}
\newcommand{\ts}[1]{\par{\color[named]{Red}TS: #1}\par}

% no labels in list of references:
\makeatletter
\renewcommand\@biblabel{}
\makeatother

\hyphenation{Snij-ders Duijn DataSpecification dataspecification dependentvariable ModelSpecification}

% centered section headings with a period after the number;
% sans serif fonts for section and subsection headings
\renewcommand{\thesection}{\arabic{section}.}
\renewcommand{\thesubsection}{\thesection\arabic{subsection}}
\makeatletter
 \renewcommand{\section}{\@startsection{section}{1}
                {0pt}{\baselineskip}{0.5\baselineskip}
                {\centering\sffamily} }
 \renewcommand{\subsection}{\@startsection{subsection}{2}
                {0pt}{0.7\baselineskip}{0.3\baselineskip}
                {\sffamily} }
 \renewcommand{\subsubsection}{\@startsection{subsubsection}{3}
                {0pt}{0.5\baselineskip}{0.3\baselineskip}
                {\it\sffamily} }
\makeatother


\renewcommand{\baselinestretch}{1.0} %% For line spacing.
\setlength{\parindent}{0pt}
\setlength{\parskip}{1ex plus1ex}
\raggedright
\newcommand{\sfn}[1]{\textbf{\texttt{#1}}}
\newcommand{\R}{{\sf R }}
\newcommand{\Rn}{{\sf R}}
\newcommand{\rs}{{\sf RSiena}}
\newcommand{\RS}{{\sf RSiena }}
\newcommand{\SI}{{\sf Siena3 }}
\newcommand{\Sn}{{\sf Siena3}}
\textheight=9.5in
\topmargin=-0.75in
\begin{document}
\algsetup{indent=2em}

\centerline{\huge{RSiena: sienaTimeTest and sienaTimeFix}}

\section{Queries}

\subsection{SienaTimeTest}
\begin{enumerate}
\item Have constructed some derivatives for ML and finite differences. Difficult
  to describe what I have done. Answers for finite differences at least seem
  sensible.
\item If we ask for specific effects to be tested, should we remove the others
  from the problem altogether or just set them not to be tested?
\item With parameter-wise tests, should this be for a group of eg eval and endow
  or for each type individually. The latter is done now, and it would be fairly
  tedious to do the former. But if it would make more sense it could be done.
\item Conditioning is always done for parameter-wise tests, but is optional for
  individual tests. This seems to be the cause of the discrepancies where there
  are three waves. Josh has a note about the possibility of discrepancies: I
  think it is because different tests were done.
\item In the print, H0 was defined as all the dummy effects being zero,
  including any which have been estimated. This seems wrong to me. And I have
  changed it to only non estimated effects.
\item All basic rates are excluded. Not clear why you could not include them in
  the estimated part of the model.
\end{enumerate}

\subsection{SienaTimeFix}

\begin{enumerate}
\item I extrapolated for endowment effects. Created an egoX for the covariate
  and set fix to TRUE unless I wanted a dummy for the density effect. Was I
  correct?
\item Not sure what to do for structural rate effects.
\item Not sure what to do for behavior effects. I have included rateX as they
  seem to be logically the same as rateX for networks. But for objective
  function effects, the density effect is accessed for networks...
\item Is it logically necessary to find the density effect for a network when
  creating dummies? Should one stop or just ignore it?
\item Would appreciate an explanation of the difference between rateX and
  objective function processing.
\end{enumerate}

\section{sienaTimeTest}

The function is called with a \nnm{sienaFit} object and performs tests for
heterogeneity of parameters over time.

\subsection{Arguments}
\begin{description}
\item[effects] Optionally, tests can be restricted to a set of effects defined
  by row number in the effects object. This should be the row number in the
  print of the \nnm{sienaFit}. (Ignoring any 0.n for conditional rates).
  This excludes all the other effects completely even from the estimated part of
  the derivative matrix. Is this correct?
\item[condition] boolean which controls whether to orthogonalise
individual score tests against just the estimated effects or against the other
unestimated terms as well. Default (FALSE) is just to use the estimated effects
in the individual tests. All are used in the parameter-wise tests i.e.\ this is
the equivalent to \emph{condition=TRUE}.
\end{description}
\subsection{Processing}
\begin{enumerate}
\item Use the \nnm{requestedEffects} element of the fit.
\item Validate as in section \ref{sec:valstt}. Stop if error.
\item Create a logical vector defining the position in the effects object of the
  effects which may be tested: \verb|baseInFit|. Details in section
  \ref{sec:baseInFit}
\item Create a logical vector defining the position in the effects object of the
  effects which will be in the top left of the Derivative matrix to be created:
  \verb|estimatedInFit| Details in section \ref{sec:estimatedInFit}
\item Make a data frame \nnm{toTest} to hold all the data needed: one row for
  each effect which may be tested and each period for which estimation was
  performed. Details of the columns is in section \ref{sec:totest}
\item Match any estimated dummies with their base effects and period in the data
  frame as described in section \ref{sec:match} and set the field
  \nnm{toTest} to FALSE in the corresponding row.
\item Overwrite the column \nnm{rowInD} of the data frame for the rows where
  \nnm{toTest} is still TRUE.  Sequentially number from one more than the number
  of rows with \nnm{toTest} equal to FALSE.
\item Sort the data frame by \nnm{rowInD}.
\item Create a new field \nnm{dummyNames} as described in section
  \ref{sec:dummyNames}
\item Create \nnm{G}, the array of deviations as in section \ref{sec:moment}
\item Create the covariance matrix \nnm{sigma} of the new moments:
covariance of sums of \nnm{G} over the second dimension (period).
\item Create \nnm{D} the derivative matrix, as described in section
\ref{sec:deriv}.
\item Calculate \nnm{fra} the mean of the deviations for each effect by summing
\nnm{G} over the first two dimensions.
\item Do Joint test: details in section \ref{sec:joint}.
\item Do Individual tests: details in section \ref{sec:indiv}
\item Do parameter-wise tests: details in section \ref{sec:par}
\item Construct a matrix for easy printing of the one step estimates. Details in
  section \ref{sec:matrix}.
\item Add more columns to the data frame \nnm{toTest} for the plot as defined in
  section \ref{sec:extra}.
\item Return a lot! Class \nnm{sienaTimeTest}.
\end{enumerate}
\subsection{Validation}\label{sec:valstt}
\begin{enumerate}
\item Check the object is of class ``sienaFit''
\item Check the estimation finished OK. (\nnm{sienaFit\$OK})
\item Must be at least two periods estimated. Get this from the length of the
\nnm{periodNos} attribute of \nnm{sienaFit\$f} because there may be groups. This
attribute has a number for each wave actually estimated, i.e. omitting the last
in each group.
\item Effect numbers requested must be numeric.
\item Effect numbers requested must be within the number of rows of the
\nnm{requestedEffects} object of the siena fit.
\item Effects specified must not be basic rate effects or dummies (i.e. have
  ``Dummy'' in their name).
\end{enumerate}
\subsection{Effects to be tested, maybe: \nnm{baseInFit}}
\label{sec:baseInFit}
\begin{enumerate}
\item If there are requested effects, exclude all others
\item Exclude any with \nnm{basicRate} TRUE
\item Exclude any where \nnm{effectName} includes the string ``Dummy''
\end{enumerate}
A check will be made for dummies already estimated for these effects: other
periods will be tested.
\subsection{Estimated Effects to be included: \nnm{estimatedInFit}}
\label{sec:estimatedInFit}
\begin{enumerate}
\item If there are requested effects, exclude all others
\item Exclude any with \nnm{basicRate} TRUE
\item Exclude any with \nnm{shortName} equal to egoX and \nnm{fix} equal TRUE
where ``Dummy'' occurs in \nnm{effectName}.
\end{enumerate}

\subsection{Details of data frame \nnm{toTest}}
\label{sec:totest}
\begin{description}
\item[baseEffect] sequential number (1 to length(indBaseEffects))
\item[effectNumber] effectNumber of base effect corresponding to the row
\item[period] \verb|periodNos| (extracted as in validation)
\item[period1] sequential number of period
\item[toTest] TRUE, initially, except for period 1, when FALSE
\item[baseRowInD] position of the basic effect corresponding to this row in D.
Used to find the right scores and deviations for the row. Find by matching
column \nnm{effectNumber} in the set of effectNumbers for the
\verb|estimatedInFit| effects.
\item[effectName] effect name of base effect corresponding to this row
\item[rowInD] position of this period and effect combination in derivative
  matrix. Fill in initially by matching effectNumber of the row with the
  effectNumbers of the \verb|estimatedInFit| effects. Values for period not eual
  to 1 will be overwritten later.
\end{description}
\subsection{Match estimated dummies with base effect}
\label{sec:match}
\begin{algorithmic}
\FORALL{effects in subset \nnm{estimatedInFit} for which \nnm{timeDummy}
contains the string ``isDummy''}
\STATE
\begin{enumerate}
\item Split the \nnm{timeDummy} field on commas. Should contain two numbers,
  first is period and second is an effectNumber.
\item Find the row in the data frame with \nnm{period} and \nnm{effectNumber}
  equal to the parts of the \nnm{timeDummy} and set \nnm{toTest} to FALSE on
  this row.
\item Match the effect number of the row against the effect numbers of the
  subset \nnm{estimatedInFit} and store it in the field \nnm{rowInD}
of this row of the data frame.
\end{enumerate}
\ENDFOR
\end{algorithmic}
\subsection{dummyNames}
\label{sec:dummyNames}
\begin{algorithmic}
\IF{toTest == FALSE}
\STATE Effect Name from the row. If type is not ``eval'',
add the type in parentheses
to make the row unique.
\ELSE
\STATE ``(*)Dummy'' plus \nnm{period} plus  \nnm{effectName} from the row.
\ENDIF
\end{algorithmic}
\subsection{Deviations array, \nnm{G}}
\label{sec:moment}
\begin{enumerate}
\item Extract the simulated statistics (or scores in maximum likelihood!)
 from the fit object \nnm{sienaFit\$sf2} for the estimated effects
 included to an array \nnm{moment} (\nnm{estimatedInFit})
\item Substract the corresponding targets from \nnm{sienaFit\$sf2}.
\item Create the \nnm{G} array with
  dimensions number of simulations (\nnm{sienaFit\$Phase3nits} by number of
  waves estimated by number of rows in the data frame \nnm{toTest}.
\item Copy \nnm{moment} to the first \nnm{nEffects}=sum(\nnm{estimatedInFit})
  slices of G (the beginning of the 3rd dimension).
\item Construct two 3-column matrices of subscripts:
  the first (destination subscripts) is formed by combining by rows
  (\textsf{rbind}) the result of
\begin{algorithmic}
\FORALL{rows to be tested}
\STATE Combine by
  columns (\textsf{cbind}): 1:number of iterations, repeated \nnm{period1},
  repeated \nnm{rowInD}
\ENDFOR
\end{algorithmic}
\item The second is similar, but replace \nnm{rowInD} by \nnm{baseRowInD}.
This gives the source subscripts.
\item Copy source entries of \nnm{moment} to destination entries in \nnm{G}
\end{enumerate}
\subsection{Derivative matrix \nnm{D}}
\label{sec:deriv}
\begin{algorithmic}
\IF{not maximum likelihood or finite difference}
\item \begin{enumerate}
\item Calculate \nnm{scores} by extracting the effects in
\nnm{estimatedInFit} from \nnm{ssc}.
\item Create the SF array with dimensions as for G
\item Copy  \nnm{scores} to the first
\nnm{nEffects}=sum(\nnm{estimatedInFit})
slices of  SF (beginning of the 3rd dimension)
\item Copy source entries of \nnm{scores} to destination entries in \nnm{SF}
using subscripts as for G.
\item Call \nnm{derivativeFromScoresAndDeviations} with parameters \nnm{G},
\nnm{SF} to calculate the derivative matrix \nnm{D}.
\end{enumerate}
\ELSE
\item Calculate the derivative matrix for maximum likelihood or
finite differences. Essentially copy rows and columns from the base, with an
extract of the original top left part at bottom right.
Once constructed, just averaged over waves and iteractions.
\ENDIF
\end{algorithmic}
\subsection{Joint tests}
\label{sec:joint}
\begin{enumerate}
\item Call \nnm{ScoreTest} with parameters \nnm{nrow(toTest), D,
sigma, fra, toTest\$toTest, maxlike=sienaFit\$maxlike}
\item Calculate joint test p-value from chisq on sum(toTest\$toTest) df,
statistic value is \nnm{testresOverall} from \nnm{ScoreTest} return value.
\item Calculate one step estimates by adding estimated theta or 0 (for tested
dummies) to the oneStep estimates returned from the joint call.
\end{enumerate}
\subsection{Individual tests}
\label{sec:indiv}
\begin{algorithmic}
\STATE Set up logical vector \nnm{doTests} with an entry for each row of
\nnm{toTests}.
\IF{condition}
\FORALL{rows of data frame with totest TRUE}
\STATE Call ScoreTest with \nnm{doTests} set to FALSE for all except this one.
All Other parameters as for joint test.
\STATE Take individual results from \nnm{testresulto} from return value from
this call.
\ENDFOR
\ELSE
\STATE Take individual results from \nnm{testresulto} from return value from
joint call.
\ENDIF
\STATE Calculate individual test p-value from normal distribution, test
statistic value is \nnm{testreso} from appropriate \nnm{ScoreTest} return
value.
\end{algorithmic}
\subsection{Parameter-wise tests}
\label{sec:par}
\begin{algorithmic}
\FORALL{distinct base effects in \nnm{toTest}}
\STATE Call ScoreTest with \nnm{doTests} set to \nnm{toTest\$toTest} for the
appropriate rows and FALSE for all other rows.
\STATE Suppress the call if all rows in the group have \nnm{toTest\$toTest}
FALSE. (This conditions always on the other effects you have not fitted.)
\ENDFOR
\STATE Calculate p-values corresponding to the results of the previous step:
chisq on sum(\nnm{totest\$toTest} for the effect) df.
\end{algorithmic}
\subsection{One step estimate matrix}
\label{sec:matrix}
Cbind
\begin{enumerate}
\item \nnm{sienaFit\$theta[estimatedInFit]} plus 0's for tested dummies
\item One step estimates as calculated from results of the joint test
\item p-values: $\hat{\theta}/\mathrm{s.e.}(\hat{\theta})$ for estimated rows
  (i.e.\ those in \nnm{estimatedInFit}) plus individual p-values for tested
  dummies calculated in the individual test section.
\end{enumerate}
Add row names from \nnm{dummyNames}.
\subsection{toTest: further columns}
\label{sec:extra}
\begin{description}
\item[InitialEst] As in one step estimate matrix.
\item[OneStepEst] As in one step estimate matrix.
\item[p.value] As in one step estimate matrix.
\item[effectTest] P-value for the parameter wise test for the base effect. NA if
  no such test performed.
\item[effectName] Add type if not eval and `\textbackslash np=' plus the
  \nnm{effectTest} value just created to the \nnm{effectName}, and recreate the
  factor, to affect the strip text in the plot.
\item[valsplus] Add \nnm{OneStepEst} (just created) for period 1 (this is the
  base effect) to corresponding one step estimates of rows for period > 1.
\item[dummysd] Calculate as follows:
\begin{description}
\item[base effect] $ \mathrm{s.e. }\hat{\theta} =
\sqrt(\mathrm{diag(sienaFit\$covtheta))}$
\item[others] Invert the field \nnm{p.value} relative to the one step estimates
  (after adding in $\hat{\theta}$).
This results in:
\begin{description}
\item[estimated dummies]
Since these p-values for the estimated dummies were calculated as
$$
\frac{\hat{\theta}}{se (\hat{\theta})}
$$
the resulting sd is
$$
\frac{se (\hat{\theta}) * \mathrm{oneStep}}{\hat{\theta}}
$$
where $\mathrm{oneStep}$ is the sum of $\hat{\theta}$ plus the one step
estimate returned from the joint test.
\item[non-estimated dummies]  Up to rounding, this is the one-step estimate
  returned from the joint test divided by the test statistic from the individual
  test.
\end{description}
\end{description}
\end{description}
\section{plot.sienaTimeTest}

There are two options here, controlled by the argument \nnm{pairwise}. In each
case a subset of the effects can be selected, using an index displayed in the
summary method.

If \nnm{pairwise} is TRUE, a pairs plot is produced of the deviations for
the base tested effects summed over waves, with the value of the correlation in
the upper quadrant.

If FALSE, there is a plot for each base effect selected, showing, for each wave,
\begin{enumerate}
\item 3 wide lines showing confidence intervals for 3 levels, centered around
  the one step estimate for the wave plus the one step estimate for the first
  wave for the base effect. (The latter has had the fitted value included.)
\item a point plot using a long character 45 (minus sign) showing the end of the
  first two levels (actually positioned at the inside of the rounded end!)
\item A stepped line for the one step estimates, with filled circles (pch 20)
at the  points.
\item A reference line across at the fitted value for the estimate.
\end{enumerate}
P-values for the overall tests of each base effect are shown (in the panel strip
by adding them to the effect name!). These are conditioned on the other dummies.

Layout can be controlled by the layout parameter of lattice. Default is to do
all the plots, with layout determined by lattice.

Size of font in strips can be changed with the lattice parameters: e.g.\
\begin{verbatim}
aa <- trellis.par.get()
aa$add.text$cex <- 0.75
trellis.par.set(aa)
\end{verbatim}
\section{sienaTimeFix}


This function is called during the initialisation phase of siena07, and
interprets entries in the \nnm{timeDummy} column of the effects object. It adds
dummy covariates to the data and effects to the effects object.

Some prior processing is done with the user entered data and effects before it
is called:  Initial values are copied from a previous result, if this is
requested.  The data is then turned into a group if it not already one.

It may also be called without a data object, when it should simply create the
new effects. This makes it possible to preview the effects which would be added
by siena07. It is an option in print.sienaEffects and summary.sienaEffects. Not
the default as you need a complete effects object for it to make sense, not just
one or two rows.
\begin{enumerate}
\item Do validation (section \ref{sec:validation})
\item If no requests or none left after validation return input unchanged
\item Replace requests for all dummies by a list of period numbers using group
periods established from the group and period numbers in the effects
object. (Unique group numbers, unique periods within groups plus 1...)
\item Remove any effects which are already dummies, defined as
 \emph{timeDummy contains the word ``isDummy''}
\item Validate the time dummy field. Stop if anything not understood.
\item Process the rateX effects with time dummy requests as in section
  \ref{sec:ratex}
\item Construct two lists of periods with non rateX dummy requests by dependent
variable. One also split by type of effect.
\item Process the objective function effects with time dummy requests as in
  section \ref{sec:obj}
\item Reorder effects so they are rate/ non interaction/interaction within
dependent variable. Dependent variables should be ordered as in the input
effects.
\item Recreate the group by calling \nnm{sienaGroupCreate}
to make sure the attributes are correct.
\end{enumerate}

\subsection{Validation}
\label{sec:validation}
\begin{enumerate}
\item If only 2 periods, all requests are an error
\item Any requests for basic rate effects are an error
\item Currently any requests for behavior effects except rateX are an error
\item Currently any requests for structural rate effects are an error
\end{enumerate}
Print warning if any errors and set the corresponding timeDummy columns to ``,''
\subsection{rateX dummies}
\label{sec:ratex}
\begin{algorithmic}
\FORALL {rateX effects with time dummy requests}
\STATE Find the covariate whose name is in \nnm{interaction1}
 in the rate effect in the lists of constant or varying covariates or
behavior variables in the attributes of the data object (group level).
\FORALL{periods for which dummies were requested}
\STATE
\begin{enumerate}
\item Construct name for new covariate:
 covariate name, ``Dummy'', period number, `:', dependent variable name.
\item Find the relevant covariate and data object for this period.
\item Make a varying covariate with the column for the current period copied
from the relevant covariate for this period. Get number of periods from the
attribute \nnm{netdims} of the dependent variable.
\item Other periods should be zeros.
\item Node Set should be copied from the covariate.
\item Add the attributes to the covariate (these would have been added by
sienaDataCreate).
\item Add to the varying covariates in the corresponding data object.
\item Also add to all the other data objects with zero value. Use the number
  of periods and the size of the nodeset from each object.
\item Create an effect as in section \ref{sec:rateeffect}.
\item Join the effect to the current effects data frame.
\end{enumerate}
\ENDFOR
\ENDFOR
\end{algorithmic}
\subsection{RateX effect}
\label{sec:rateeffect}
\begin{enumerate}
\item Use the function \nnm{CreateEffects} to create an effect with effectGroup
  equal to ``covarNonSymmetricRate'' and short name equal to ``RateX''with name,
  type, netType, groupName, group taken from the base effect.
\item Other values: \begin{description}
\item[fix] FALSE
\item[include] TRUE
\item[effectNumber] current maximum + 1
\item[timeDummy] isDummy, period number, effectNumber of base effect
\end{description}
\item Set row name of row to the name of the new covariate
\end{enumerate}
\subsection{Objective function effect dummies}
\label{sec:obj}
\begin{algorithmic}
\FORALL{dependent variables with dummies requested}
\IF{behavior variable}
\STATE  stop with error message (currently trapped in validation!)
\ENDIF
\FORALL{periods for which dummies were requested}
\STATE Create dummy covariates and egoX effects. See section \ref{sec:objcov}
\ENDFOR
\STATE
\FORALL{periods for which dummies were requested}
\STATE Create interaction effects. See section \ref{sec:objint}

\ENDFOR
\ENDFOR
\end{algorithmic}
\subsection{Objective function dummy covariates and dummy effects}
\label{sec:objcov}
\begin{enumerate}
\item Find the relevant data object and hence number of actors and periods. Use
  attribute \nnm{netdims} to avoid complications with sparse matrices.
\item Convert period number to relative to start of the current data object.
\item Create a varying covariate with name Dummy followed by the period number
  plus `:' and the name of the dependent variable. Values are 1 for the relevant
  period and 0 otherwise.
\item Add the attributes to the covariate (which would have been added by
sienaDataCreate)
\item Add to the varying covariates in the correct data object.
\item Add a zero valued version of this dummy covariate to all other data
objects.
\item Look for the density effects for this dependent variable for any types
for which any dummies were requested for this variable and establish
whether a time dummy was requested for these effects for this type for
this period. Stop if none found.
\item Create effect(s) as in section \ref{sec:nonrate}
\item Join the effect(s) to the current effects data frame
\end{enumerate}
\subsection{Objective function interaction effects}
\label{sec:objint}
\begin{enumerate}
\item Use includeInteraction to include an interaction between the dummy effect
for this period and the current requested effect.
\item Parameters are the short name of the requested effect, ``egoX'',
interaction1 (second entry) is the new dummy, type from the requested effect,
and name from the current dependent variable, interaction1 (first entry) and
interaction2 (first entry) from the requested effect.
\item Use character=TRUE to avoid problems.
\item Set the time dummy for this row to ``isDummy, period, effectnumber of
requested effect.''
\end{enumerate}
\subsection{Objective function dummy effect}
\label{sec:nonrate}
\begin{enumerate}
\item The base effects are the density effects. Create row(s) with effectGroup
equal to ``covarNonSymmetricObjective'' and short name equal to ``egoX''. Name,
netType, groupName, group, type taken from the base effect(s).
\item Other values:
\begin{description}
\item[fix] FALSE unless a time dummy for the density effect for this type
has been requested for this period.
\item[include] FALSE unless a time dummy has been requested for this type for
  this period
\item[effectNumber] current maximum + 1:(nunber of effects being created)
\item[timeDummy] isDummy, period number, effectNumber of density effect for this
  type
\end{description}

\item Set row name of row to the name of the new covariate and the type,
  separated by a `.'.
\end{enumerate}
\end{document}

