\documentclass[12pt, a4paper]{article}
%\usepackage[pdftex,dvipsnames]{color}
\usepackage{graphicx}
\usepackage{times}
\usepackage{algorithm}
\usepackage{algorithmic}
\usepackage{url}
\usepackage{amsmath}
\usepackage{amsfonts}

\usepackage[pdfstartview={},
            pdftex,
            bookmarks=true,
            bookmarksopen=true,
            bookmarksnumbered=true,
            ]{hyperref}

\newcommand{\eps}{\varepsilon}
\newcommand{\abso}[1]{\;\mid#1\mid\;}
\renewcommand{\=}{\,=\,}
\newcommand{\+}{\,+\,}
% ----------------------------------------------------------------
\newcommand{\remark}[1]{\par\noindent{\color[named]{ProcessBlue}#1}\par}
\newcommand{\mcc}[2]{\multicolumn{#1}{c}{#2}}
\newcommand{\mcp}[2]{\multicolumn{#1}{c|}{#2}}
\newcommand{\nm}[1]{\textsf{\small #1}}
\newcommand{\nnm}[1]{\textsf{\small\textit{#1}}}
\newcommand{\nmm}[1]{\nnm{#1}}
\newcommand{\sfn}[1]{\textbf{\texttt{#1}}}
\newcommand{\bs}{\backslash}

\newcommand{\rs}{{\sf RSiena}}
\newcommand{\RS}{{\sf RSiena }}
\newcommand{\rst}{{\sf RSienaTest}}
\newcommand{\RST}{{\sf RSienaTest }}

% no labels in list of references:
\makeatletter
\renewcommand\@biblabel{}
\makeatother

\hyphenation{Snij-ders Duijn DataSpecification dataspecification
  dependentvariable ModelSpecification}

\makeatother


\renewcommand{\baselinestretch}{1.0} %% For line spacing.
\setlength{\parindent}{0pt}
\setlength{\parskip}{1ex plus1ex}
\raggedright
\oddsidemargin 0in
\evensidemargin 0in
\textwidth 6.25in
\textheight 9.9in
\topmargin -2cm
\begin{document}

\title{\vspace*{-0.5cm} Notes for Committing RSiena to R-Forge}
\author{Tom Snijders}
\date{}
\maketitle

\centerline{\emph{\today}}
\bigskip

This is a short set of guidelines for committing the RSiena and RSienaTest packages
to R-Forge. It is an excerpt of the \emph{Notes for Developers of RSiena}
(file \texttt{RSienaDeveloper}). More complete information can be found there.
Both of these files are in the directory \texttt{RSienaTest{$\bs$}doc}
in the package source.

It is wise to have a look at \emph{Notes for Developers of RSiena}.
For anything in this document that you don't understand, it may be illuminating.
Further, its section on personal/private keys was useful to me.


\section{Basics}

The connection to R-Forge is made by TortoiseSVN.
To check and compile the package, you need the most recent version of R
and RTools. Note that RTools must be in the path of your computer.

\section{The code; correspondence between packages}

\emph{Notes for Developers of RSiena} mentions coding standards for R as well as C++.
Try to respect them.
Add comments in your code and use helpful names (in camel case).

Keep the line endings in all \texttt{.tex} files intact (i.e., do not let
an editor take away the hard returns).

The correspondence between \RS and \RST should be kept intact:
\begin{enumerate}
\item Both packages should have the same version number, \texttt{ChangeLog}, and
     manual; this means that if (e.g.) you commit a new version of \RST only,
     you also need to change the version number and \texttt{ChangeLog} of \rs,
     and see to it that the files \texttt{RSiena\_Manual.tex},\ \texttt{RSiena\_Manual.pdf},
     and \texttt{RSiena.bib} in directory \texttt{$\bs$inst$\bs$doc} are identical
     in both packages.
\item Preferably, the number of the R-Forge revision (e.g., 276) should
     be the same as the last number in the \RS version number (e.g., 1.1-276).
     Evidently, this will lead to questions when moving to version 1.2-xxx etc.
     We shall deal with those when the time comes.
\item The moment when things are moved from \RST to \RS is up to the programmers.
     Some things, certainly error corrections, should be changed in both simultaneously.
     When we are unsure about a change in the code, it may be better to first
     make it in \RST and then decide later, after having obtained more experience,
     about moving it to \rs.
\item For aligning files in \RS with those in \RST I make use (on my Windows machine)
     of WinMerge (it's free and useful).
\end{enumerate}

The internal structure of both packages should remain intact:
\begin{enumerate}
\item Change the version number in the \texttt{DESCRIPTION} file and in
      \texttt{$\bs$man$\bs$RSiena-package.Rd};
      add your changes to the top of the \texttt{ChangeLog} and,
      in so far as they are noticeable
      to users, to the top of appendix B in the manual.
\item When new effects are added, these \underline{must} be added to
      Chapter 12 of the manual.
\item Also new other functionalities, if any, should be added to the manual,
      unless they are clear enough from the help pages.
\end{enumerate}

To keep things easy for users of different platforms, when you add a file to the system,
in TortoiseSVN set the property \sfn{svn:eoln-style} to ``native".
(This is not necessary for .pdf files.)
This will avoid the all-too-usual problems when transferring files between Windows and Mac systems.\\
Recipe: right click on the file name $\rightarrow$ TortoiseSVN $\rightarrow$ Properties  $\rightarrow$
New  $\rightarrow$ EOL  $\rightarrow$ Platform-dependent (native).\\
Remember to do this also for the identical file that you create for the second package (e.g., for \RS if
first you added the file to \rst).

Preferably limit line lengths to a maximum of 80 characters.

\section{Checking}


 It will be much appreciated when you update the description files in
 the \text{doc} directory, or add new files, in cases where you made changes that are
 important enough to deserve such documentation.

  Before committing, it is mandatory to either run \\
  \verb|R CMD check RSiena|\\
  , or do a build, and then run \\
  \verb|R CMD check RSiena_1.1-xxx.tar.gz|\\
  ; do the same for \rst.
  Use the most recent version of R for this (note that
  successive versions and subversions of R often increase the stringency of checking).
  Note that the C++ compiler may be different on different systems (e.g., Windows and Mac)
  which may lead to different warnings.

  Try to make the code so that notes and warnings from this Check are avoided as far as possible,
  and that you understand those that remain.
  There should of course be no errors at all.
  For example, I always get a note that the package size is big
  (because the manual is a very big file) and
  often about some superfluous pieces of code (because I want to use them later).

  Checks consist of running the examples in the help files (unless excluded by \texttt{dontrun})
  and in \texttt{$\bs$tests}, except those that are excluded in the file \texttt{.Rbuildignore}.
  This currently leaves only \texttt{$\bs$tests$\bs$parallel.R}.
  Sometimes the file \texttt{$\bs$tests$\bs$parallel.Rout.save} must be updated.

  It may be useful to add other scripts for checking (but they should not consume much time).

\end{document}
