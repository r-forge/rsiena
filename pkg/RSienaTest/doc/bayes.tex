\documentclass[12pt,a4paper]{article}
\usepackage[pdftex,dvipsnames]{color}
\usepackage{graphicx,times}
\usepackage{amsmath}
\usepackage{amsfonts}
\usepackage{amssymb}
\usepackage{algorithm}
\usepackage{algorithmic}
\usepackage[pdfstartview={}]{hyperref}
\textheight=9.5in
\textwidth=6.0in
\topmargin=-0.5in
\evensidemargin=0in
\oddsidemargin=0in
\newcommand{\eps}{\varepsilon}
\newcommand{\abso}[1]{\;\mid#1\mid\;}
\renewcommand{\=}{\,=\,}
\newcommand{\+}{\,+\,}
% ----------------------------------------------------------------
\newcommand{\remark}[1]{\par\noindent{\color[named]{ProcessBlue}#1}\par}
\newcommand{\mcc}[2]{\multicolumn{#1}{c}{#2}}
\newcommand{\mcp}[2]{\multicolumn{#1}{c|}{#2}}
\newcommand{\nm}[1]{\textsf{\small #1}}
\newcommand{\nnm}[1]{\textsf{\small\textit{#1}}}
\newcommand{\nmm}[1]{\nnm{#1}}
\newcommand{\R}{{\sf R }}
\newcommand{\sfn}[1]{\textbf{\texttt{#1}}}
\newcommand{\Rn}{{\sf R}}
\newcommand{\rs}{{\sf RSiena}}
\newcommand{\RS}{{\sf RSiena }}
\newcommand{\SI}{{\sf Siena3 }}
\newcommand{\Sn}{{\sf Siena3}}
% no labels in list of references:
\makeatletter
\renewcommand\@biblabel{}
\makeatother

\hyphenation{Snij-ders Duijn DataSpecification dataspecification dependentvariable ModelSpecification}

% centered section headings with a period after the number;
% sans serif fonts for section and subsection headings
\renewcommand{\thesection}{\arabic{section}.}
\renewcommand{\thesubsection}{\thesection\arabic{subsection}}
\makeatletter
 \renewcommand{\section}{\@startsection{section}{1}
                {0pt}{\baselineskip}{0.5\baselineskip}
                {\centering\sffamily} }
 \renewcommand{\subsection}{\@startsection{subsection}{2}
                {0pt}{0.7\baselineskip}{0.3\baselineskip}
                {\sffamily} }
\makeatother

\newcommand{\ts}[1]{\par{\color[named]{Red}TS: #1}\par}

\renewcommand{\baselinestretch}{1.0} %% For line spacing.
\setlength{\parindent}{0pt}
\setlength{\parskip}{1ex plus1ex}
\begin{document}

\title{RSiena: Bayesian models}
\author{Ruth Ripley \& Tom Snijders}
\date{}
\maketitle

\centerline{\emph{\today}}
\bigskip
\emph{Initialization}
\begin{enumerate}
\item Estimate using MoM with multigroup option, i.e., under assumption
    of common parameter values, using only 2 subphases: \nnm{startupGlobal}
\item Using this as a starting value, for each group separately
    do 0 or 2 subphases to get initial values \nnm{initialEstimates} per group,
    and covariance matrices \nnm{proposalCov} for proposal distributions
    per group.
\item Basic rate parameters are now on sqrt scale. This can be easily modified
    by changing functions \nnm{trafo}, \nnm{antitrafo}, \nnm{devtrafo}.
\end{enumerate}
\paragraph{Main algorithm}
\begin{enumerate}
  \item Set up data in C as usual
  \item Create minimal chain and do burnin
  \item \nnm{improveMH}: Get scalefactors such that about 40 out of 100
  Bayesian proposals after single MH steps are accepted.
  See below for details of
      generation and probabilities for Bayesian proposals. Keep theta
      unchanged throughout this step. \label{item5}
\item Do a warming phase of \nnm{nwarm} iterations of some number of MH steps.
\item Repeat step \ref{item5}.
\item Do requested number of Bayesian iterations. The
  length of the ML ones are determined by the multiplication factor and the
  observed distance.
\end{enumerate}
\paragraph{Bayesian proposals}
\label{sec:prop}
\begin{algorithmic}
\FORALL {groups}
\STATE Create a mask to exclude basic rate effects for other periods than this
group.
\STATE Get a multivariate normal with mean 0 and  \nnm{proposalCov}
 $\times$ scale factor for this group
\STATE Calculate the proposal probability:
\begin{description}
\item[prior] Multivariate normal density for the parameters with mean 0
  and covariance as supplied in the input argument.
\item[chain] Add
\begin{enumerate}
\item sum of log probabilities of choice of variable/actor
\item sum of log choice probabilities
\item minus the sum of basic rate parameters times the
relevant number of actors
\item sum of log(basic rate) parameter times the
number of real steps in the chain for the corresponding variable.
\end{enumerate}
(If not constant rates, use mu and sigma from the normal approximation instead.)
Since chain does not change size, ignore the log factorial of chain length.
\end{description}
\STATE The log probability of acceptance is then new - old  of log prior +
log chain
\ENDFOR
\end{algorithmic}
\end{document}
