\documentclass[12pt,a4paper]{article}
\usepackage[pdftex,dvipsnames]{color}
\usepackage{graphicx,times}
\usepackage{amsmath}
\usepackage{amsfonts}
\usepackage{amssymb}
\usepackage{algorithm}
\usepackage{algorithmic}
\usepackage[pdfstartview={}]{hyperref}
\textheight=9.5in
\textwidth=6.0in
\topmargin=-0.5in
\evensidemargin=0in
\oddsidemargin=0in
\newcommand{\eps}{\varepsilon}
\newcommand{\abso}[1]{\;\mid#1\mid\;}
\renewcommand{\=}{\,=\,}
\newcommand{\+}{\,+\,}
% ----------------------------------------------------------------
\newcommand{\remark}[1]{\par\noindent{\color[named]{ProcessBlue}#1}\par}
\newcommand{\mcc}[2]{\multicolumn{#1}{c}{#2}}
\newcommand{\mcp}[2]{\multicolumn{#1}{c|}{#2}}
\newcommand{\nm}[1]{\textsf{\small #1}}
\newcommand{\nnm}[1]{\textsf{\small\textit{#1}}}
\newcommand{\nmm}[1]{\nnm{#1}}
\newcommand{\R}{{\sf R }}
\newcommand{\sfn}[1]{\textbf{\texttt{#1}}}
\newcommand{\Rn}{{\sf R}}
\newcommand{\rs}{{\sf RSiena}}
\newcommand{\RS}{{\sf RSiena }}
\newcommand{\SI}{{\sf Siena3 }}
\newcommand{\Sn}{{\sf Siena3}}
% no labels in list of references:
\makeatletter
\renewcommand\@biblabel{}
\makeatother

\hyphenation{Snij-ders Duijn DataSpecification dataspecification dependentvariable ModelSpecification}

% centered section headings with a period after the number;
% sans serif fonts for section and subsection headings
\renewcommand{\thesection}{\arabic{section}.}
\renewcommand{\thesubsection}{\thesection\arabic{subsection}}
\makeatletter
 \renewcommand{\section}{\@startsection{section}{1}
                {0pt}{\baselineskip}{0.5\baselineskip}
                {\centering\sffamily} }
 \renewcommand{\subsection}{\@startsection{subsection}{2}
                {0pt}{0.7\baselineskip}{0.3\baselineskip}
                {\sffamily} }
\makeatother

\newcommand{\ts}[1]{\par{\color[named]{Red}TS: #1}\par}

\renewcommand{\baselinestretch}{1.0} %% For line spacing.
\setlength{\parindent}{0pt}
\setlength{\parskip}{1ex plus1ex}
\begin{document}

\title{RSiena: Bayesian models}
\author{Ruth Ripley}
\date{}
\maketitle

\centerline{\emph{\today}}
\bigskip
\begin{enumerate}
\item I inverted dfra on advice from Brian.
\item I use log scale for basic rate parameters: I struggled with transforming
  the derivatives. What I currently have is due to Brian, but I may well have
  misunderstood.
\item Input of a covariance matrix for the prior is mandatory: default the
  identity times 10000.
\item Input of a covariance matrix for the Bayesian proposal is optional: if
  null it will be calculated from a few iterations of the MH step.
\item Default $n$, number of iterations for this step if performed is 10.
\item dfra may be extracted along with theta from a previous result of siena07
via prevAns.
\end{enumerate}
\paragraph{Main algorithm}
\begin{enumerate}
  \item Set up data in C as usual
  \item Create minimal chain and do burnin
  \item Do $n$ groups of \nnm{nrunMH} (as calculated by siena07 using the
    multiplcation factor and the observed distance) steps to
  estimate \nnm{dfra} as in siena07.
  \item Transform \nnm{dfra} so can use basic rate parameters on log scale:
Multiply rows and columns for basic rate parameters by the value. Diagonal
entries for basic rate parameters should be
$$ \textrm{deriv} * \textrm{value} + \textrm{score} * \textrm{value}^2 $$
  \item Get scalefactors such that about 40 out of 100 Bayesian proposals after
  single MH steps are accepted. See below for details of generation and
  probabilities for Bayesian proposals. Keep theta unchanged throughout this
  step.
\item Do a warming phase of \nnm{nwarm} iterations of some number (hard-coded to
  4 at the moment) of MH steps.
\item Do requested number of Bayesian iterations. The
  length of the ML ones are determined by the multiplication factor and the
  observed distance.
\end{enumerate}
\paragraph{Bayesian proposals}
\label{sec:prop}
\begin{algorithmic}
\FORALL {groups}
\STATE Create a mask to exclude basic rate effects for other periods than this
group.
\STATE Get a multivariate normal with mean 0 and covariance masked inverse of
\nnm{dfra} * scale factor for this group
\STATE Calculate the proposal probability:
\begin{description}
\item[prior] Multivariate normal density for the parameters with mean 0
  and covariance as supplied in the input argument.
\item[chain] Add
\begin{enumerate}
\item sum of log probabilities of choice of variable/actor
\item sum of log choice probabilities
\item minus the sum of basic rate parameters times the
relevant number of actors
\item sum of log(basic rate) parameter times the
number of real steps in the chain for the corresponding variable.
\end{enumerate}
(If not constant rates, use mu and sigma from the normal approximation instead.)
Since chain does not change size, ignore the log factorial of chain length.
\end{description}
\STATE The log probability of acceptance is then new - old  of log prior +
log chain
\ENDFOR
\end{algorithmic}
\end{document}
