\documentclass[a4paper,fleqn,12pt]{article}
% The parameter in [..] can be beamer or handout
% Based on file
% $Header: /cvsroot/latex-beamer/latex-beamer/solutions/generic-talks/generic-ornate-15min-45min.en.tex,v 1.4 2004/10/07 20:53:08 tantau Exp $
% This file is a solution template for:
% - Giving a talk on some subject.
% - The talk is between 15min and 45min long.
% - Style is ornate.
% Copyright 2004 by Till Tantau <tantau@users.sourceforge.net>.

% When replacing figures, put a % at the end of the line
% to prevent a linebreak being read!



\usepackage{pgf}
%\usepackage[dvipsnames]{xcolor}
\usepackage{tikz}
\usepackage{pgflibraryarrows}
\usepackage{colortbl}
\usepackage{amsmath}
\usepackage{latexsym}
\usepackage{enumitem}

% For handouts; see pgfmanual
%\usepackage{pgfpages}
%\pgfpagesuselayout{2 on 1}[border shrink=5mm]


\usepackage{times}
\usepackage[T1]{fontenc}
% Or whatever. Note that the encoding and the font should match. If T1
% does not look nice, try deleting the line with the fontenc.

% If you have a file called "university-logo-filename.xxx", where xxx
% is a graphic format that can be processed by latex or pdflatex,
% resp., then you can add a logo as follows:


\newcommand{\Reals}{\mbox{I}\hspace{-.07ex}\mbox{R}}
\newcommand{\E}{\mbox{E}}
\renewcommand{\P}{\mbox{P}}
\newcommand{\se}{\mbox{s.e. }}
\newcommand{\var}{\mbox{var}}
\newcommand{\Cov}{\mbox{Cov}}
\newcommand{\logit}{\mbox{logit}}
\newcommand{\uh}{\underline{h}}
\newcommand{\tuh}{\tilde{\uh}}
\newcommand{\neqsum}[3]
{\, \sum_{\stackrel{\scriptstyle #1 = 1}{\scriptstyle #2 \neq #3}}^g
\,}

\newcommand{\equat}[1]{$#1$}
\newcounter{savenumi}
\newcounter{savenumi2}
\newcounter{savenumi3}
\newcounter{savenumi4}





% If you wish to uncover everything in a step-wise fashion, uncomment
% the following command:

%\beamerdefaultoverlayspecification{<+->}
\renewcommand{\baselinestretch}{1.1}
\newcommand{\cc}{\color[named]{SeaGreen}}
\newcommand{\itc}{\color[rgb]{0.1,0.6,0.3}}  % itemcolor
\newcommand{\siec}{\color[named]{Sienna}} % Siena color (RawSiena dvipsnames)
\newcommand{\emec}{\color[rgb]{0.3,0.1,0.0}}  %  \color[named]{Emerald}}


\newcommand{\cdiamond}{{\color[named]{MidnightBlue}\diamond}}
\newcommand{\cbullet}{{\color[named]{MidnightBlue}$\bullet$}}
\newcommand{\cques}{{\color[named]{MidnightBlue}?`}}
\newcommand{\rast}{{\color[named]{Red}$\ast$}}
\newcommand{\gast}{{\color[named]{Green}$\ast$}}

\newcommand{\rtar}{{\itc $\Rightarrow$}}
\newcommand{\qques}{\raisebox{1.2ex}{\rotatebox{180}
                    {\color[named]{ForestGreen}\bf\large ?}}}
\newcommand{\excl}{{\color[named]{ForestGreen}\bf\large !}}
% or OliveGreen

\hyphenation{mini-step mini-steps reci-pro-city}

\setlength{\parskip}{1.5ex plus0.5ex minus0.5ex}
\setlength{\parindent}{0ex}

\title{Siena Chain Structures}

\author{Tom A.B.\ Snijders }




\newcommand{\sx}{{\scriptstyle X}}
\newcommand{\sz}{{\scriptstyle Z}}
\newcommand{\extraver}{{$\phantom{\Big(}$}}
\newcommand{\extraverr}{{$\phantom{\big(}$}}

\renewcommand{\th}[1]{$\theta_{#1}$}
\newcommand{\ga}[1]{$\gamma_{#1}$}
\newcommand{\be}[1]{$\beta_{#1}$}
\newcommand{\maxr}{\textit{maxbeh}_r}

\newcommand{\vit}{\theenumi}

\newcommand{\mcc}[2]{\multicolumn{#1}{c}{#2}}
\newcommand{\mcp}[2]{\multicolumn{#1}{c|}{#2}}

\newcommand{\equa}[1]{\[#1\]}

\newcommand{\separationb}{\\[0.5ex]\hline\rule{0pt}{2ex}}
\newcommand{\separationg}{\\[0.5ex]\arrayrulecolor{grey}\hline\arrayrulecolor{black}
                         \rule{0pt}{2ex}}

\newcommand{\nm}[1]{\textsf{\small #1}}
\newcommand{\nnm}[1]{\textsf{\small\textit{#1}}}
\newcommand{\nmm}[1]{\nnm{#1}}
\newcommand{\R}{{\sf R }}
\newcommand{\Rn}{{\sf R}}
\newcommand{\ms}{\textsl{\textsf{\small ms}}} % ministep

\begin{document}

\maketitle

This paper gives a specification of the
chain structures for likelihood-based calculations in RSiena.
Much of this was implemented in Siena 3 in the unit \textsf{chains}.
However, the procedures in this unit do much more than necessary,
and in some cases do things slightly differently.

\section{Notation}


Symbols given \nnm{in italic sf font} refer to the names
of variables used in the \R code.

I changed the earlier used letters $X$ and $Z$ referring to networks and behavior
to $N$ and $B$.

Not all notation presented is used in this note;
I just copied the notation from \emph{Siena Algorithms}
and added what was needed additionally.

\emph{Generic symbols for variables}

There are $R_N$ networks and $R_B$ behavior variables.\\
We require $R_N + R_B \geq 1$; if the current structure of RSiena
requires this, then we require $R_N \geq 1$ (but if it is easy to
work with $R_N = 0, R_B \geq 1$ then it would be nice to permit this.)

\begin{tabbing}
$i, j$ \hspace*{1em} \=  \hspace*{6em} \=  actors.\\[1ex]
$m$ \> \> index for time period from $t_{m-1}$ to $t_m$ ($m = 2, \ldots, M$).  \\[1ex]
$M$ \> \nnm{observations}   \> total number of observations\\[1ex]
$x$ \> \> all $R_N$ networks jointly (one outcome).\\[1ex]
$z$ \> \> all $R_B$ behaviors jointly (one outcome).\\[1ex]
$y$ \> \> state: all networks and behaviors jointly (one outcome).\\[1ex]
$W$ \> \> variable with values $N$ or $B$, indicating \\
    \> \> whether something refers to network or behavior.\\[1ex]
$r$ \> \> index number of networks or behaviors,\\
    \> \> ranging from 1 to $R_W$ .\\[1ex]
missing \> \> missingness indicators \\
        \> \>  for ordered triples $(i,j,r)$ referring to networks $r$ \\
        \> \> and for ordered pairs $(i,r)$ referring to behaviors;\\
        \> \> values are F (false) and T (true);\\
        \> \> if T, then further specifications are Start / End / Both,\\
        \> \> referring to the observation period from $t_{m-1}$ to $t_m$ .\\[1ex]
$\maxr$ \> \> maximum of the range of the $r'^{text{th}}$ behavior variable.\\[1ex]
$^N$ \> \> as superscript: refers to network dynamics.\\[1ex]
$^B$ \> \> as superscript: refers to behavior dynamics.\\[1ex]
\end{tabbing}
\medskip

\emph{Changing variables (outcomes)}

\begin{tabbing}
$^{\rm{obs}}$ \hspace*{1em} \=  \hspace*{6em} \= as superscript: refers to observed values.\\[1ex]
$\theta$ \> \nnm{theta} \> vector of all statistical parameters. \\[1ex]
$p$  \>  \nnm{pp}  \> dimension of $\theta$.\\[1ex]
$J$ \> \>  simulated data score function (vector of partial derivatives of log-likelihood) \\
     \>  \>    ($p$-vector).\\[1ex]
$t$ \hspace*{3em} \> \> time.\\[1ex]
$N^{(r)}_{ij}$ \> \> dummy tie variable indicating $ i \stackrel{r}{\rightarrow} j $ for $r^{\text{th}}$ network.\\[1ex]
$B^{(r)}_{i}$ \> \> behavior variable for $r^{\text{th}}$ behavior for actor $i$.
\end{tabbing}
\medskip

Replacing an index by + denotes summation over this index.\\
Toggling a number $a$ in $\{0, 1\}$ means replacing $a$ by $1-a$.
\medskip

\emph{Functions}

\begin{tabbing}
$\lambda^N(r,i,x,z)$ \hspace*{3em} \= rate function, network $r$.\\
                                  \> (0 for inactive actors)\\[1ex]
$\lambda^B(r,i,x,z)$ \hspace*{3em} \> rate function, behavior $r$.\\
                                  \> (0 for inactive actors)\\[1ex]
$f^N(r,i,x,z)$ \hspace*{3em} \> evaluation function function, network $r$.\\[1ex]
$f^B(r,i,x,z)$ \hspace*{3em} \> evaluation function, behavior $r$.\\[1ex]
$g^N(r,i,j,x,z)$ \hspace*{3em} \> endowment function function, network $r$.\\[1ex]
$g^B(r,i,x,z)$ \hspace*{3em} \> endowment function, behavior $r$.\\[1ex]
$\Delta f^N(r,i,j,x,z)$ \hspace*{3em} \> change in $f^N(r,i,x,z)$
                                by toggling $N^{(r)}_{ij}$.\\[1ex]
$\Delta f^B(r,i,v,x,z)$ \hspace*{3em} \> change in $f^B(r,i,x,z)$
                                by changing $B^{(r)}_i$ to $B^{(r)}_i + v$.
%$ \sim E(\lambda)$ \> generate random variable according to exponential distribution\\
%                   \> with parameter $\lambda$ (note: expected value $1/\lambda$).\\[1em]
\end{tabbing}
%\medskip

\section{Data structures}


The basic data structure is called a \emph{chain}.
This is a sequence of changes that can take one (`observed') value
of $y$ to a next one.

To allow later generalization to valued networks as easily as possible,
we define a condition $D$ (for dichotomous) that is defined on the
level of variables (networks or behavioral variables);
in our current system $D$ is true for networks and false for
behavioral variables, but this can be different in future uses.

One change is called a \nnm{ministep}, denoted \ms, and is defined as:
\begin{equation}
    \ms \ = (w,i,j,r,d,\textit{pred},\textit{succ},
                    \textit{lprob}, \textit{rrate})    \label{ministep}
\end{equation}
where
\begin{tabbing}
$w$ (`aspect') $\phantom{abcdefghij}$ \= = \= `network' or `behavior' (abbreviated to $N$ -- $B$ );\\
$i$ (`actor') \> = \> actor if $w$ = B, sending actor if $w$ = N;\\
$j$ (`actor') \> = \> meaningless 0 if $w$ = B, receiving actor if $w$ = N;\\
$r$ (`variable number') \> = \> number of variable ($1 \leq r \leq R_w$ );\\
$d$ (`difference') \> = \> meaningless 0 if $D$, amount of change if not $D$\\
   \> \>   (where $D $ depends on $w, r$);\\
   \> \>    currently we require $d \in \{-1, 0, 1\}$, but at some \\
   \> \>    later moment exceptions to this rule may be allowed;\\
\textit{pred} (`predecessor') \> = \> pointer to preceding ministep;\\
\textit{succ} (`successor') \> = \> pointer to next (succeeding) ministep;\\
\textit{lprob} (`log probability') \> = \> log probability of making this ministep;\\
\textit{rrate} (`reciprocal rate') \> = \> reciprocal of aggregate (summed) )rate function \\
   \> \>      immediately before this ministep.
\end{tabbing}
To indicate the components/fields of a ministep we use the notation
$\ms.w, \ms.i$, etc.

In Siena 3, $d$, \textit{pred} and \textit{succ} are called \textit{difh},
\textit{predh} and \textit{such};
and the program uses rates instead of reciprocal rates, but this was
implemented only very incompletely anyway.

The ministep is practically the same as what is called a microstep
in the paper `Siena Algorithms', but used here in a more
precise way. These words are not intentionally different.
\\
The log probability and reciprocal rate depend not only on the
chain and the ministep, but also on the initial state $y$ or $y(t_{m-1})$
valid before the start of the chain; and on the model specification and
model parameters.
Their computation is done by procedure \textit{StepProb} described in
Section~\ref{S_prob}.

The interpretation is that a ministep operates on (i.e., changes)
outcome $y$ as implemented by the following function.
\begin{enumerate}
\item \textit{ChangeStep}$(y, \ms)$ transforms state $y$ as follows,\\
      where $ \ms \ = (w,i,j,r,d, ...)$:
      \begin{itemize}
      \item if $w$ = N and $i \neq j$, change $N^{(r)}_{ij}$ to $1 - N^{(r)}_{ij}$;
      \item if $w$ = B,  change $B^{(r)}_{i}$ to $B^{(r)}_{i} + d$.
      \end{itemize}
\end{enumerate}
This means that only those values of $d$ are allowed that do not lead
$B^{(r)}_{i}$ outside of the bounds of this variable.
I think this should not always be checked except perhaps for in a test phase,
but the creation and transformation of ministeps should contain
checks that ensure this condition.

\textit{ChangeStep} is called a lot, and it will be helpful
that it is implemented in a very fast way.

The \emph{kind} of a ministep is defined as
(Network, $i,\, j,\, r$) for Network ministeps,
and as (Behavior, $i, \, r$) for Behavior ministeps.
This defines the variable changing by the ministep.

The \emph{subkind} of a ministep is defined as
$(w,i,r)$.
This defines the choice situation / option set for the ministep.

This definition also means that network ministeps with $i = j$ and behavior ministeps
with $d = 0$ have no effect on the outcome. Such ministeps are permitted,
and are called \emph{diagonal} ministeps.

A \emph{chain} from observation $y(t_{m-1})$
to observation $y(t_m)$ is a sequence of ministeps $\ms_1 , \ms_2 , ..., \ms_T$
which, when applied sequentially, transform $y(t_{m-1})$ into $y(t_{m})$.
We then say that the chain \emph{connects}  $y(t_{m-1})$ to $y(t_m)$.
For $M$ observations, therefore, we require a sequence of $M-1$ chains.

For a sequence of ministeps $\ms_1 , \ms_2 , ..., \ms_T$
we define the following functions.
For disregarded values of the ministep (depending on whether it is a N or B
ministep) we use the wildcard symbol *.
\begin{enumerate}[resume]
\item \textit{NetworkNumber}$(i,j,r,t)  =  \,  \sharp\{ s \mid 1 \leq s \leq t, \textit{kind}(\ms_s) = (N,i,j,r)  \} $
\item \textit{BehSum}$(i,B,r,t)  = \,   \Sigma_{s=1}^t \, \ms_s.d_s\,
                      I\{ \textit{kind}(\ms_s) = (B,i,*,r) \} $\\
      where $I$ is the indicator function defined as $I(A) = 1$ if $A$ is True
      and 0 if $A$ is False.
\end{enumerate}

If the outcomes $y(t_{m-1})$ and $y(t_m)$ are completely defined
(without any missing data) then the requirements on this sequence are as follows.

\emph{Networks} : (since changes are defined as toggles)\\
For all $i, j, r$ with $1 \leq r \leq R_N$, $i \neq j$,
\[
N^{(r)}_{ij}(t_{m-1}) = N^{(r)}_{ij}(t_m) \ \Leftrightarrow \
       \text{\textit{NetworkNumber}}(i,j,N,r,T) \text{ is even }; \label{netrec}
 \]
\emph{Behavior} : (since changes are defined as increments)\\
For all $i, r$ with $1 \leq r \leq R_B$,
\begin{eqnarray}
B^{(r)}_{i}(t_{m-1}) \! &+&  \! \text{ \textit{BehSum}}(i,B,r,T) = B^{(r)}_{i}(t_m)  \label{behrec1} \\
\text{and} &&  \nonumber \\
1 \leq B^{(r)}_{i}(t_{m-1}) \!  &+&  \! \text{ \textit{BehSum}}(i,B,r,t) \leq
                             \maxr   \text{ for all } 1 \leq t < T.   \label{behrec2}
\end{eqnarray}


For missing network data $(i,j,r)$ and missing behavior data $(i,r)$,
there is no such requirement; although for missing behavior data
it must be ensured that the variable remains within range.

It must be noted that missing data are not handled in the best possible way
in the likelihood-based procedures in Siena 3, and this will be changed
in RSiena. Therefore, results for likelihood-based procedures in Siena 3 and RSiena
will be different.

Classes of functions are required which do the following:
\begin{enumerate}
\item Create and transform chains.
\item Calculate probabilities related to chains.
\item Store chains: read from and write to file.
\end{enumerate}

\section{Create and transform chains}

\subsection{Data types}
\begin{enumerate}
\item  Ministep. See (\ref{ministep}).\\
       The \emph{kind} of a ministep is (Network, $i,\, j,\, r$) for Network ministeps,
       and (Behavior, $i, \, r$) for Behavior ministeps.\\
       Note that in Siena 3 this is called the \emph{rkind} (\emph{restricted kind}),
       and the `kind' there also includes the value $d$.
       A ministep \ms \ is \emph{diagonal} if it is of kind (Network, $i,\, j,\, r$)
       with $i = j$, or of kind (Behavior, $i, \, r$) with $\ms.d$ = 0.
\item Chain. This is a sequence of ministeps connected by the pointers
      \textit{pred} and \textit{succ},
      with a first and last element.
      The first and last elements are dummies, i.e., they are of a special kind
      $(\textit{Extreme}, 0, 0, 0)$ which implies no change:\\
      $\textit{ChangeStep}(y,\text{first}) = \textit{ChangeStep}(y,\text{last}) = y$.\\
      The section on structurally fixed values gives an exception to this rule,
      however, for the last element.\\
      The first and last elements are used just to have handles
      for the start and end of the chain. Of course, first.\textit{pred} =last.\textit{succ} = nil.\\
      The first and last ministeps are not diagonal.\\
      \\
      The connection implies that if $\ms_a$ and $\ms_b$ are two ministeps
      with \\
      $\ms_b.pred = \ms_a$, then $\ms_a.succ = \ms_b$.
      The first element has a nil \textit{pred}, and the last element
      has a nil \textit{succ}.
\end{enumerate}

\subsection{Functions}

In Siena 3, I have defined the ministep type with various other pointers and attributes
useful for navigating in the chain.
These are functions of the chain, and including them in the ministep type
is for the purpose of computational efficiency.
These functions are the following. They are defined as functions of the ministep
in a given chain.

\begin{enumerate}
\item \textit{nrkind}. The total number of ministeps in the chain of the same kind.
\item \textit{predkind}. Pointer to the last earlier (`preceding') ministep of the same kind.\\
                         Called \textit{predhrkind} in Siena 3.
\item \textit{succkind}. Pointer to the first later (`succeeding') ministep of the same kind.\\
                         Called \textit{suchrkind} in Siena 3.
%\item \textit{multiple}. A boolean attribute, indicating that there is at least one more
%      ministep of the same kind. The first and last ministeps are not multiple.
\end{enumerate}

The chain defines an order relation (binary function) of ministeps
in an obvious way -- which of the ministeps comes earlier.
When there may be the possibility of confusion,
this is called the \emph{chain order}.

\begin{enumerate}
\item $\ms_a < \ms_b$ of there is a sequence $\ms_1, ..., \ms_K$ ($K \geq 0$) of ministeps
      such that $\ms_a.succ = \ms_1, \ms_b.pred = \ms_k$, and $\ms_k.succ = \ms_{k+1}$
      for all $k$, $1 \leq k \leq K-1$.
\end{enumerate}

An ordered pair of ministeps is called a CCP (\emph{consecutive canceling pair})
if they are of the same kind, not diagonal, cancel each other's effect (see next sentence),
have no other ministep of the same kind in between, are ordered
according to the chain order, and there is at least one ministep of a
different kind in between.
Two ministeps $\ms_a$ and $\ms_b$ cancel each other's effect
if the following hold: either they are both of the same kind
(Network, $i,\, j,\, r$) (then they cancel because they toggle the same
binary variable), or both are of the same kind (Behavior, $i, \, r$)
and $\ms_a.d + \ms_b.d = 0$.

For example, if the chain contains a total of three ministeps
$\ms_a, \ms_b, \ms_c$
of the kind (Network, 1, 2, 1), with $\ms_a < \ms_b < \ms_c$,
and none of which are each others' immediate predecessors/successors,
then $(\ms_a, \ms_b)$ and $(\ms_b, \ms_c)$ are CCP's.

Basic functions of the chain are the following.
\begin{enumerate}
%\item \textit{Valid}, a boolean indicating whether the chain connects
%      $y(t_{m-1})$ to $y(t_m)$, and the log probabilities and rates
%      have been calculated.
\item \textit{TotNumber}, the number of ministeps of the chain, excluding the first;\\
      so an empty chain consisting only of the first and last ministeps
      with \\
      first.\textit{succ} = last has $TotNumber = 1$.
\item \textit{DiagNumber}, the number of diagonal ministeps of the chain.
\item\textit{CCPNumber}, the number of CCP's in the chain.
\end{enumerate}

The following functions are defined on the kinds of ministeps:
\begin{enumerate}
\item \textit{NumberKind}$(w,i,j,r)$, the number of ministeps of the chain
      of kind $(w,i,j,r)$.\\
      For the network ministeps this will be a sparse matrix, in the sense that for large networks
      most of the values \textit{NumberKind}$(N,i,j,r)$ will be 0.\\
      This is called \textit{Numberrkind} in Siena 3.
\item \textit{Multiple}$(w,i,j,r)$ = True if \textit{NumberKind}$(w,i,j,r) \geq 2$
      and False if \textit{NumberKind}$(w,i,j,r) \leq 1$.\\
      We say that a kind can be multiple or non-multiple.
\end{enumerate}

\subsection{Operations}

Basic operations on chains are the following.
Of course they have to guarantee the consistency of all the
derived variables and pointers.
The consistency of the log probabilities and reciprocal rates
is treated separately (when it is needed), see Section~\ref{S_prob}.
\begin{enumerate}
\item \emph{Create} an empty chain consisting only of the elements (first, last).
%      with \textit{Valid} = False.
%\item \emph{Validate}, check that the chain connects $y(t_{m-1})$ to $y(t_m)$
%      and if so, calculate the log probabilities and rates
%      and set \textit{Valid} to True.
\item \emph{InsertBefore}$(\ms, (w,i,j,d,r))$ :\\
      for a currently existing ministep $\ms \neq $  first, insert the ministep
      with values $(w,i,j,d,r)$ between \ms.\textit{pred} and \ms.
\item \emph{Delete}  a ministep, and link up its predecessor and successor.
\item \emph{RandomElement}      :\\
      draw a random ministep from the chain, excluding the first element;
      note that the probabilities are 1/\textit{TotNumber}.
\item \emph{Connect} : construct randomly a chain that connects
      two outcomes $y(t_{m-1})$ and $y(t_m)$ .\\
      This is done by repeatedly applying \emph{RandomElement} and \emph{InsertBefore}:
      \begin{tabbing}
      For all \= $R_N$ networks:\\
         \> For all \= $(i,j), i \neq j$:\\
          \> \> if $N^{(r)}_{ij}(t_{m-1}) \neq N^{(r)}_{ij}(t_m)$,\\
            \> \> then InsertBefore(RandomElement, $N,i,j,0,r$);\\
      For all \= $R_B$ behaviors: \\
         \> For all \= $i$:\\
          \> \> Define $D = B^{(r)}_{i}(t_{m}) - B^{(r)}_{i}(t_{m-1})$;\\
            \> \> if $D > 0$, then $D$ times InsertBefore(RandomElement, $B,i,0,1,r$);\\
            \> \> if $D < 0$, then $-D$ times InsertBefore(RandomElement, $B,i,0,-1,r$).\\
      EXPLAIN LATER: mh DIAGONAL INSERTS UNTIL K'TH REJECTION.
      \end{tabbing}
\end{enumerate}
The following random draws are not always possible, since the sets
from which a random element is drawn, may be empty.
(It may be noted, however, that usually the set will be non-empty.)
\begin{enumerate}[resume]
\item \emph{RandomDiag}      :\\
      draw a random diagonal ministep from the chain;
      note that the probabilities are 1/\textit{DiagNumber}.
\item \emph{RandomMultipleKind}      :\\
      draw a random multiple kind;
      This is called \textit{RandomMultipleS} in Siena 3,
      but there returns a ministep of this kind rather than
      the kind itself.
%      note that the probabilities are 1/\textit{MultipleNumber}.
\item \emph{RandomCCPKind}$(w,i,j,r)$    :\\
      draw a random element from the set of all ministeps in the chain
      that are the first element of some CCP of kind $(w,i,j,r)$;
      note that the probabilities are 1/\textit{DiagNumber}.\\
      This is called RandomCCPrkind in Siena 3.\\
      I SHALL WRITE MORE HERE ABOUT THE POINTERS USED IN SIENA 3
      TO DO THIS.
\end{enumerate}

\section{Calculate probabilities related to chains}
\label{S_prob}

Ministeps are interpreted as changes in the chain (procedure \textit{ChangeStep}).
These changes are made with certain probabilities, and the
rate of change has a certain value when the ministep is going to be made.
The probabilities and rates depend on the state immediately before the ministep;
this depends in turn on the state at the start of the chain, and the
sequence of ministeps before the current ministep.
For a ministep \ms \ in the chain with a given initial state $y$
(say, $y = y(t_{m-1})$), the state obtaining before \ms \
can be defined recursively as follows (where \textit{ChangeStep} is treated
as a function with states as outcomes).
\begin{enumerate}
\item $\textit{StateBefore}(\text{first}) = y$.
\item $\textit{StateBefore}(\ms.succ) = \textit{ChangeStep}(\textit{StateBefore}(\ms),\ms)$.
\end{enumerate}
Thus, the state before \ms \ is obtained by repeatedly applying \textit{ChangeStep}.

The probabilities and rates are defined by the following procedure.
For the mathematical symbols $\pi_j, \pi_v, \lambda(...), J_m$, see
the notation of Section 2 of `Siena Algorithms' where the microstep/ministep
is treated for the purpose of simulation of the model,
and where the same ingredients are used.
\begin{enumerate}
\item \textit{StepProb}$(\textsf{input}\ y, w,i,j,r,d; \textsf{output}\ lpr, rr, sc)$;\\
      this calculates for the current state $y$, conditional on the
      assumption that a ministep of kind $(w,i,*,r)$ is made,
      the conditional probability that this will be the ministep
      with value $(w,i,j,r,d)$;
      this is $\pi_j$ (for networks) and $\pi_v$ (for behavior,
      but now with $v$ replaced by $d$), denoted here by $\pi$;\\
      it calculates the probability to make
      a ministep of subkind $(w,i,r)$, which is
      $\lambda^w(r,i,y) / \lambda^+(+,+,y)$;\\
      the output value
      $lpr = \ln(\pi) + \ln(\lambda^w(r,i,y)) - \ln(\lambda^+(+,+,y))$
      is the log of the product of these probabilities;\\
      it calculates the aggregate rate for current state $y$ ;
      this is given by $\lambda^+(+,+,y)$; output value
      $rr = 1/\lambda^+(+,+,y)$ is the reciprocal of this rate;\\
      and it calculates the contribution of this ministep to the
      score function, which in Section 2 of `Siena Algorithms'
      is what is added to $J_m$, and stores this in $sc$.
\end{enumerate}
Function \textit{StepProb} in Siena 3 calculates $\ln(\pi)$ rather than what
is defined here as $lpr$. I think that a more straightforward algorithm
is obtained by calculating the definition of $lpr$ given here.

The values of \textit{lprob} and \textit{rrate} that are part of the definition
of the ministeps must be as calculated by \textit{StepProb} when applied to the
state immediately before the ministep.
Thus, in practice, these values can be obtained for the whole chain
by repeated application of \textit{StepProb} and \textit{ChangeStep} jointly.

While operating on the chain, it is important to keep the
log probabilities and rates up to date. This requires the following procedure.
It updates only part of the chain, and is applied when it is known
that the earlier and later parts do not need to be updated.

\begin{enumerate}[resume]
\item \textit{Update}($\ms_a, \ms_b$) for ministeps $\ms_a < \ms_b$:\\
       Use \textit{StepProb} to update the log probabilities and rates for all ministeps
       from $\ms_a$ to $\ms_b$ (i.e., all ministeps between these two in the chain order,
       including these two ministeps themselves).\\
       This is called \textit{UpdateRateslprobs} in Siena 3.
\end{enumerate}


\section{Metropolis-Hastings steps}

A basic required functionality is to simulate from the distribution
of chains that connect $y(t_{m-1})$ to $y(t_m)$,
given the model specification and model parameters.
This is done by repeated application of Metropolis Hastings steps.
These are of the following three types.

\begin{enumerate}
\item \textit{MH\_InsDelPermute}
\item \textit{MH\_InsertDiag}\\
      (called \textit{MH\_TryInsertDiag} in Siena 3).
\item \textit{MH\_CancelDiag}\\
      (called \textit{MH\_TryCancelDiag} in Siena 3).
\end{enumerate}

\subsection{Diagonal Insert}

The function \textit{MH\_InsertDiag}$(\textsf{input}\ \ms, w,i,r; \textsf{output}\ \textit{accept})$
is defined as follows.
The interpretation is that,
the proposal is made to insert a diagonal element of subkind
$(w,i,r)$ immediately before ministep \ms;
according to a random decision with a certain probability
this proposal is put into effect (yielding \textit{accept} = True)
or not (yielding \textit{accept} = False).

FURTHER THIS TEXT IS INCOMPLETE, TO BE ADDED TO.

\section{Store chains}

For communication with users and with other programs,
it is necessary to have a way of reading chains from files
and writing them to files.
Chains also have to be communicated to R.

\section{Structurally fixed values}

If $N^{r}(i,j)$ is structurally fixed and $N^{r}(i,j)(t_{m-1}) = N^{r}(i,j)(t_m)$,
then the chain for period $m$ must not contain any ministeps of kind (Network, $i,j,r$).\\
If $B^{r}(i)$ is structurally fixed and $B^{r}(i)(t_{m-1}) = B^{r}(i)(t_m)$,
then the chain for period $m$ must not contain any ministeps of kind (Behavior, $i,*,r$).

For the variables that are structurally fixed but
have values at $t_m$ different from their
values at $t_{m-1}$, the principle is that these changes are enforced
either directly before the last ministep, or as part of the last ministep
(whichever is the simpler or more elegant;
I think the latter). These changes
do not contribute anything to probabilities or rates; this can be implemented
formally by omitting them from sums or by defining \textit{lprob} = 0
and \textit{rrate} = 0.
If there are several such variables, the order in which these changes are enforced
does not matter (and is inconsequential).


\end{document}
