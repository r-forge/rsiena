\PassOptionsToPackage{colorlinks=true}{hyperref}
\documentclass{tufte-handout}
\usepackage[utf8]{inputenc}
\usepackage{amsmath,amssymb}

\usepackage{listings}
% https://tex.stackexchange.com/questions/282485/use-listings-in-tufte-book-with-captions-in-margin
\makeatletter
% enable re-use of \listoflistings facility
\setboolean{@tufte@packages@subfigure}{true}
\def\ext@lstlisting{lol}
% textwidth Tuftian float for listings
\newenvironment{listing}[1][htbp]
  {\ifvmode\else\unskip\fi\begin{@tufte@float}[#1]{lstlisting}{}}
  {\end{@tufte@float}}
% fullwidth Tuftian float for listings
\newenvironment{listing*}[1][htbp]%
  {\ifvmode\else\unskip\fi\begin{@tufte@float}[#1]{lstlisting}{star}}
  {\end{@tufte@float}}
% show listing number in caption even though \lst@@caption is empty
\def\fnum@lstlisting{\lstlistingname~\thelstlisting}
\makeatother
\def\linestyle{\color{white!50!black}}
\lstset{
% escapechar=\%,
% mathescape=true, % bad idea with $var scripting languages
escapeinside={<@}{@>},
language=, % empty language == plain text
backgroundcolor=,
basicstyle=\scriptsize\ttfamily,
keywordstyle=\bfseries\color{black},
identifierstyle=\color{black},
commentstyle=\color{darkgray},
stringstyle=\color{red},
showstringspaces=false,
numbers=left,
numberstyle=\linestyle,
numbersep=1.6ex,
keepspaces=true, % non-leading white space
showtabs=true,
basewidth=1.1ex,
xleftmargin=2em, %\parindent,
literate={-}{-}1%
{|}{|}1%
{<}{<}1{>}{>}1%
{~}{\raisebox{-0.55ex}[0pt][-1ex]{\~{}}}1%
,
}

\title{Tutorial: Creating an Effect}
\author{Felix Schönenberger}

% links to the right place if the file is viewed in the doc folder
% TODO open local file links platform independent?
\def\linkstyle{} % TODO condensed?
\newcommand\sourcelinkPF[2]{\href{../#1/#2}{\linkstyle{#1/}{#2}}}
\newcommand\sourcelinkpF[2]{\href{../#1/#2}{\linkstyle{#2}}}
\newcommand\sourcelinkpFF[3]{\href{../#1/#2}{\linkstyle{#2::}{#3}}}
\newcommand\sourcelinkpfF[3]{\href{../#1/#2}{\linkstyle{#3}}}

\begin{document}
\maketitle

\begin{abstract}
This is a hands on tutorial creating a new effect and adding it to RSiena.
It links other sources of documentation (when available) or the relevant source files.
\end{abstract}

\section{Effect Implementation}

Effects are organized in an hierarchical inheritance structure\footnote{See \url{./classdesign.pdf} for attempts at a more detailed class rundown. However code usually is the most up-to-date, accessible, but vast and obscure source of information.} which roughly follows the effect synopsis.
For this tutorial we are recreating the \emph{gmm} type reciprocity effects for \emph{new}, \emph{real} and \emph{persistent} reciprocity.
\begin{align}\label{eq:def}
  s_{\text{recip}}(x) &= \sum_i \sum_j \left\{\begin{array}{ll}
     1 & \text{if } x^{t}_{ij}+x^{t}_{ji}=k \wedge x^{t+1}_{ij}+x^{t+1}_{ji}=2\\
     0 & \text{otherwise}
  \end{array}\right.
\end{align}
with $k=0$ for new, $k=1$ for real and $k=2$ for persistent reciprocity.

Reciprocity only depends on the network, we therefore inherit from \sourcelinkpF{src/model/effects}{NetworkEffect.h} (listing~\ref{lst:class}).

\begin{listing}
\label{lst:class}
\caption{
  \sourcelinkpF{src/model/effects/gmm}{ReciprocityGMMEffect.h}: Structure of the effect.
}
% using <@\label{line:const}@> line number are off
\begin{lstlisting}[language=c++]
class ReciprocityGMMEffect: public NetworkEffect {
public:
  const static int NEW = 0;
  const static int REAL = 1;
  const static int PERSISTENT = 2;

  ReciprocityGMMEffect(const EffectInfo * pEffectInfo, const int startEdgeSum) :
    NetworkEffect(pEffectInfo)
  {
    this->startEdgeSum = startEdgeSum;
  }

  double calculateContribution(int) const { return 0; }
protected:
  virtual double tieStatistic(int alter) { /* effect code */ }
private:
  int startEdgeSum;
}
\end{lstlisting}
\end{listing}
\begin{description}\itemsep0pt
\item[\linestyle line 3-5] Integer constants to keep the math generic.
\item[\linestyle line 7-11] Each effect constructor must take an \texttt{EffectInfo} pointer and can take any number of optinal arguments.
\item[\linestyle line 13] Being a gmm effect only the effect value is needed at the end of a simulation run, \sourcelinkpfF{src/model/effects}{NetworkEffect.h}{calculateContribution} will never be called.
\item[\linestyle line 15] Will hold the effect code.
\end{description}

The \sourcelinkpF{src/model}{StatisticCalculator.h} is responsible for initializing and evaluating effects outside of the simulation\footnote{
  During the simulation only the contribution to a change is needed.
  Evaluation happens in the responsible Variable e.g. \sourcelinkpfF{src/model/variables}{NetworkVariable.h}{calculateTieFlipContribution}
}.
Top level effect classes split the calculation into meaningful defaults\footnote{
  See \sourcelinkpfF{src/model/effects}{NetworkEffect.h}{statistic} and look at the default implementation of the methods it calls.
}.
For most effects it will suffice to implement \sourcelinkpfF{src/model/effects}{NetworkEffect.h}{tieStatistic} and leave the two outer loops as is.

\begin{align*}
  s_{\text{recip}}(x) &= \sourcelinkpfF{src/model/effects}{NetworkEffect.h}{statistic}(x) \\
  s_{\text{recip}}(x) &= \sum_{i \in V} \sourcelinkpfF{src/model/effects}{NetworkEffect.h}{egoStatistic}(x,i) \\
  s_{\text{recip}}(x) &= \sum_{i \in V} \sum_{j \in V} \sourcelinkpfF{src/model/effects}{NetworkEffect.h}{tieStatistic}(x,i,j)
\end{align*}

\begin{listing}
\caption{
  Implementation of equation \ref{eq:def}.
}
\begin{lstlisting}[language=c++]
  virtual double tieStatistic(int alter) {
    // Outgoing tie exists, otherwise we would not be here.
    // Check if it is reciprocated.
    if (this->inTieExists(alter)) {
      // We have reciprocity in the end state.
      // Now look at the start of the period.
      const Network* pStart = this->pData()->pNetwork(this->period());
      if (pStart->tieValue(this->ego(), alter)
          + pStart->tieValue(alter, this->ego()) == startEdgeSum) {
            return 1;
      }
    }
    return 0;
  }
\end{lstlisting}
\end{listing}
\begin{description}\itemsep0pt
\item[\linestyle line 4] Check if the reciprocal tie exists in the current time point\footnote{
    This might not access the network directly but can use a cached store.
    See all the super classes to familiarize yourself with what is available.
  }.
\item[\linestyle line 7] Get the network at the beginning of the period.
\item[\linestyle line 8-11] Check the first part of equation \ref{eq:def} directly on the network.
\end{description}

\section{Binding the Effect to R}

Lets have a look at a \emph{typical life of an effect} and inject the required snippets.

\def\linecontinoue{\textcolor{red}{\ensuremath{\hookrightarrow}}}
\begin{enumerate}
\item \sourcelinkpF{data}{allEffects.csv} is parsed constructing a gigantic table.
Lines are expanded/rejected based on the actual data object used.
For reciprocity we limited the data to directed networks (encoded in the csv as \texttt{nonSymmetricObjective}).
Listing \ref{lst:alleff} shows the header line and entries for the gmm reciprocity effect family.
\begin{listing*}
\label{lst:alleff}
\caption{
  \texttt{allEffects.csv}: Each line (\linecontinoue continoues a line) is one effect definition for a data scenario.
}
\begin{lstlisting}[language=,numbers=none]
effectGroup,effectName,functionName,shortName,endowment,interaction1,interaction2,type,basicRate,include,randomEffects,fix,test,
<@$\linecontinoue$@>timeDummy,initialValue,parm,functionType,period,rateType,untrimmedValue,effect1,effect2,effect3,interactionType,local
\end{lstlisting}
\begin{lstlisting}[language=,numbers=none]
nonSymmetricObjective,new recip.,new reciprocated ties,newrecip,FALSE,,,gmm,FALSE,FALSE,FALSE,FALSE,FALSE,
<@$\linecontinoue$@>",",0,0,gmm,NA,NA,0,0,0,0,dyadic,FALSE
nonSymmetricObjective,persistent recip.,persistent rec. ties,persistrecip,FALSE,,,gmm,FALSE,FALSE,FALSE,FALSE,FALSE,
<@$\linecontinoue$@>",",0,0,gmm,NA,NA,0,0,0,0,dyadic,FALSE
nonSymmetricObjective,real recip.,real reciprocated ties,realrecip,FALSE,,,gmm,FALSE,FALSE,FALSE,FALSE,FALSE,
<@$\linecontinoue$@>",",0,0,gmm,NA,NA,0,0,0,0,dyadic,FALSE
\end{lstlisting}
\end{listing*}

This tasks is best accomplished by searching for and adapting the entries of a similar effect.

\item \sourcelinkpFF{R}{sienaeffects.r}{includeEffect} flags the effect as included.

\item \sourcelinkpFF{R}{initializeFRAN.r}{initializeFRAM} constructs a vector of \sourcelinkpF{src/model}{EffectInfo.h} objects via \sourcelinkpFF{src}{siena07setup.cpp}{effects}.

\item Effects are instantiated by \sourcelinkpF{src/model/effects}{EffectFactory.h}.
To register an effect add is header to the includes at the top\footnote{
  The \sourcelinkpF{src/model/effects}{AllEffects.h} attempts to list all effects.
} and a conditional to the \texttt{createEffect} function \ref{lst:factory}.

\begin{listing}
\label{lst:factory}
\caption{
  \sourcelinkpF{src/model/effects}{EffectFacttory.h} consisting mainly of one big conditional.
}
\begin{lstlisting}[language=c++]
Effect * EffectFactory::createEffect(const EffectInfo * pEffectInfo) const {
  /* huge if-else block with other effects... */
  if (effectName == "newrecip") {
    pEffect = new ReciprocityGMMEffect(pEffectInfo, ReciprocityGMMEffect::NEW);
  } else if (effectName == "realrecip") {
    pEffect = new ReciprocityGMMEffect(pEffectInfo, ReciprocityGMMEffect::REAL);
  } else if (effectName == "persistrecip") {
    pEffect = new ReciprocityGMMEffect(pEffectInfo, ReciprocityGMMEffect::PERSISTENT);
  }
  /* ...huge if-else block with other effects */
}
\end{lstlisting}
\end{listing}
\end{enumerate}

\section{Deriving non-cross-lagged gmm effects}

This section affects cross-lagged effects inheriting from \sourcelinkpF{src/model/effects}{CovariateDependentNetworkEffect.h} for which non-cross-lagged gmm effects should be implemented.
See \sourcelinkpF{src/model/effects}{CovariateSimilarityEffect.h} for a live implementation.

\paragraph{Infrastructure:}
\sourcelinkpfF{src/model}{StatisticCalculator.h}{calculateNetworkGMMStatistics} calls \sourcelinkpfF{src/model}{CovariateDependentNetworkEffect.h}{initialize} with two states, the normal period begin state and the simulation state.  The constructor takes a \texttt{simulatedState} flag to switch the behaviour.
\begin{description}
\item[\texttt{simulatedState=false}] Unchanged behaviour.
\item[\texttt{simulatedState=true}] Change the behaviour access in the following ways:
\begin{itemize}
\item Variable values are initialized from the simulated state.
\item Covariates are access from the end of the period not the start.
\end{itemize}
\end{description}
This affects all pointer and functions in the class.
However and effect could still opt out and access the data directly in which case additional work has to be done.

\paragraph{Derive an effect:}
With that in place a non-cross-lagged effect can be instantiated by passing on \texttt{simulatedState=true} to the super constructor and of course registering it as new effect\footnote{
  Once Viviana gives her OK we can patch all the effects inheriting from \sourcelinkpF{src/model/effects}{CovariateDependentNetworkEffect.h} in one go.
}.

\section{Function types}

The model distinguishes 5 different effect functions.
Each function has its own evaluation mechanism in the \sourcelinkpF{src/model}{StatisticCalculator.h} and some functions in the variable classes.
Some effects do not fulfill the contract of functions which is captured only in the \texttt{allEffects} database and validated in R.

\begin{description}
\item[\sourcelinkpfF{src/model}{Model.h}{rEvaluationEffects}]  ...
\item[\sourcelinkpfF{src/model}{Model.h}{rCreationEffects}] Can hold the same effects as the evaluation function.
  SC: Effects are initialize with a modified state object constructed by subtracting the start of the period.
  DV: \textcolor{red}{difference in dependent variable?}
\item[\sourcelinkpfF{src/model}{Model.h}{rEndowmentEffects}] Similar to the creation function.
  SC: Effects are initialize with a modified state object with only deleted ties.
    DV: \textcolor{red}{difference in dependent variable?}
\item[\sourcelinkpfF{src/model}{Model.h}{rRateEffects}] ...
\item[\sourcelinkpfF{src/model}{Model.h}{rGMMEffects}] Effect vector for the gmm estimator.
Effects in this function are only evaluated outside of simulations and do not need to calculate contributions.
\end{description}
Effects actually have different \sourcelinkpfF{src/model/effects}{NetworkEffect.h}{creationStatistic} and \sourcelinkpfF{src/model/effects}{NetworkEffect.h}{endownmentStatistic} functions.
However the default implementation of those link to the evaluation statistic and all rely on the modified network state fed by the calculator.

\section{TODO/Thoughts}

\paragraph{explain the contribution part a bit}

\paragraph{hint to other than network sides}

\begin{itemize}
\item full list of sane top level entry points
\end{itemize}

\paragraph{difference between eval/endown/maintain implementation}

\paragraph{best practice for effects}

\begin{itemize}
\item h / cpp
\item grouping
\item cache usage
\end{itemize}

\paragraph{testing}

\begin{itemize}
\item integration
\end{itemize}

\end{document}
