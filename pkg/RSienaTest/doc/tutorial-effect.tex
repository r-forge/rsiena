\PassOptionsToPackage{colorlinks=true}{hyperref}
\documentclass{tufte-handout}
\usepackage[utf8]{inputenc}
\usepackage{amsmath,amssymb}

\usepackage{listings}
% https://tex.stackexchange.com/questions/282485/use-listings-in-tufte-book-with-captions-in-margin
\makeatletter
% enable re-use of \listoflistings facility
\setboolean{@tufte@packages@subfigure}{true}
\def\ext@lstlisting{lol}
% textwidth Tuftian float for listings
\newenvironment{listing}[1][htbp]
  {\ifvmode\else\unskip\fi\begin{@tufte@float}[#1]{lstlisting}{}}
  {\end{@tufte@float}}
% fullwidth Tuftian float for listings
\newenvironment{listing*}[1][htbp]%
  {\ifvmode\else\unskip\fi\begin{@tufte@float}[#1]{lstlisting}{star}}
  {\end{@tufte@float}}
% show listing number in caption even though \lst@@caption is empty
\def\fnum@lstlisting{\lstlistingname~\thelstlisting}
\makeatother
\def\linestyle{\color{white!50!black}}
\lstset{
% escapechar=\%,
% mathescape=true, % bad idea with $var scripting languages
escapeinside={<@}{@>},
language=, % empty language == plain text
backgroundcolor=,
basicstyle=\scriptsize\ttfamily,
keywordstyle=\bfseries\color{black},
identifierstyle=\color{black},
commentstyle=\color{darkgray},
stringstyle=\color{red},
showstringspaces=false,
numbers=left,
numberstyle=\linestyle,
numbersep=1.6ex,
keepspaces=true, % non-leading white space
showtabs=true,
basewidth=1.1ex,
xleftmargin=2em, %\parindent,
literate={-}{-}1%
{|}{|}1%
{<}{<}1{>}{>}1%
{~}{\raisebox{-0.55ex}[0pt][-1ex]{\~{}}}1%
,
}

\title{Tutorial: Creating an Effect}
\author{Felix Schönenberger}

% links to the right place if the file is viewed in the doc folder
% TODO local file links?
\newcommand\sourcelinkF[2]{\href{../#1/#2}{\mbox{#1/}\mbox{#2}}}
\newcommand\sourcelinkFF[3]{\href{../#1/#2}{\mbox{#1/}\mbox{#2::#3}}}
\newcommand\sourcelinkfF[3]{\href{../#1/#2}{\mbox{#2::#3}}}

\begin{document}
\maketitle

\begin{abstract}
This is a hands on tutorial creating a new effect and adding it to RSiena.
It links other sources of documentation (when available) or the relevant source files.
\end{abstract}

\section{Effect Implementation}

Effects are organized in an hierarchical inheritance structure\footnote{See \url{./classdesign.pdf} for attempts at a more detailed class rundown. However code usually is the most up-to-date, accessible, but vast and obscure source of information.} which roughly follows the effect synopsis.
For this tutorial we are recreating the \emph{gmm} type reciprocity effects for \emph{new}, \emph{real} and \emph{persistent} reciprocity.
\begin{align}\label{eq:def}
  s_{\text{recip}}(x) &= \sum_i \sum_j \left\{\begin{array}{ll}
     1 & \text{if } x^{t}_{ij}+x^{t}_{ji}=k \wedge x^{t+1}_{ij}+x^{t+1}_{ji}=2\\
     0 & \text{otherwise}
  \end{array}\right.
\end{align}
with $k=0$ for new, $k=1$ for real and $k=2$ for persistent reciprocity.

Reciprocity only depends on the network, we therefore inherit from \sourcelinkF{src/model/effects}{NetworkEffect.h} (listing~\ref{lst:class}).

\begin{listing}
\label{lst:class}
\caption{
  \sourcelinkF{src/model/effects/gmm}{ReciprocityGMMEffect.h}: Structure of the effect.
}
% using <@\label{line:const}@> line number are off
\begin{lstlisting}[language=c++]
class ReciprocityGMMEffect: public NetworkEffect {
public:
  const static int NEW = 0;
  const static int REAL = 1;
  const static int PERSISTENT = 2;

  ReciprocityGMMEffect(const EffectInfo * pEffectInfo, const int startEdgeSum) :
    NetworkEffect(pEffectInfo)
  {
    this->startEdgeSum = startEdgeSum;
  }

  double calculateContribution(int) const { return 0; }
protected:
  virtual double tieStatistic(int alter) { /* effect code */ }
private:
  int startEdgeSum;
}
\end{lstlisting}
\end{listing}
\begin{description}\itemsep0pt
  \item[\linestyle line 3-5] Integer constants to keep the math generic.
  \item[\linestyle line 7-11] Each effect constructor must take an \texttt{EffectInfo} pointer and can take any number of optinal arguments.
  \item[\linestyle line 13] Being a gmm effect only the effect value is needed at the end of a simulation run, \sourcelinkfF{src/model/effects}{NetworkEffect.h}{calculateContribution} will never be called.
  \item[\linestyle line 15] Will hold the effect code.
\end{description}

The \sourcelinkF{src/model}{StatisticCalculator.h} is responsible for initializing and evaluating effects outside of the simulation
\footnote{
  During the simulation only the contribution to a change is needed.
  Evaluation happens in the responsible Variable e.g. \sourcelinkfF{src/model/variables}{NetworkVariable.h}{calculateTieFlipContribution}
}.
Top level effect classes split the calculation into meaningful defaults
\footnote{
  See \sourcelinkfF{src/model/effects}{NetworkEffect.h}{statistic} and look at the default implementation of the methods it calls.
}.
For most effects it will suffice to implement \sourcelinkfF{src/model/effects}{NetworkEffect.h}{tieStatistic} and leave the two outer loops as is.

\begin{align*}
  s_{\text{recip}}(x) &= \sourcelinkfF{src/model/effects}{NetworkEffect.h}{statistic}(x) \\
  s_{\text{recip}}(x) &= \sum_{i \in V} \sourcelinkfF{src/model/effects}{NetworkEffect.h}{egoStatistic}(x,i) \\
  s_{\text{recip}}(x) &= \sum_{i \in V} \sum_{j \in V} \sourcelinkfF{src/model/effects}{NetworkEffect.h}{tieStatistic}(x,i,j)
\end{align*}

\begin{listing}
\caption{
  Implementation of equation \ref{eq:def}.
}
\begin{lstlisting}[language=c++]
  virtual double tieStatistic(int alter) {
    // Outgoing tie exists, otherwise we would not be here.
    // Check if it is reciprocated.
    if (this->inTieExists(alter)) {
      // We have reciprocity in the end state.
      // Now look at the start of the period.
      const Network* pStart = this->pData()->pNetwork(this->period());
      if (pStart->tieValue(this->ego(), alter)
          + pStart->tieValue(alter, this->ego()) == startEdgeSum) {
            return 1;
      }
    }
    return 0;
  }
\end{lstlisting}
\end{listing}
\begin{description}\itemsep0pt
  \item[\linestyle line 4] Check if the reciprocal tie exists in the current time point\footnote{
      This might not access the network directly but can use a cached store.
      See all the super classes to familiarize yourself with what is available.
    }.
  \item[\linestyle line 7] Get the network at the beginning of the period.
  \item[\linestyle line 8-11] Check the first part of equation \ref{eq:def} directly on the network.
\end{description}

\section{Binding the Effect to R}

Lets have a look at a \emph{typical life of an effect} and inject the required snippets.

\def\linecontinoue{\textcolor{red}{\ensuremath{\hookrightarrow}}}
\begin{enumerate}
\item \sourcelinkF{data}{allEffects.csv} is parsed constructing a gigantic table.  Lines are expanded/rejected based on the actual data object used.
\begin{listing*}
\caption{
  \texttt{allEffects.csv}: Each line (\linecontinoue continoues a line) is one effect definition for a data scenario.
}
\begin{lstlisting}[language=,numbers=none]
effectGroup,effectName,functionName,shortName,endowment,interaction1,interaction2,type,basicRate,include,randomEffects,fix,test,
<@$\linecontinoue$@>timeDummy,initialValue,parm,functionType,period,rateType,untrimmedValue,effect1,effect2,effect3,interactionType,local
nonSymmetricObjective,new recip.,new reciprocated ties,newrecip,FALSE,,,gmm,FALSE,FALSE,FALSE,FALSE,FALSE,
<@$\linecontinoue$@>",",0,0,gmm,NA,NA,0,0,0,0,dyadic,FALSE
nonSymmetricObjective,persistent recip.,persistent rec. ties,persistrecip,FALSE,,,gmm,FALSE,FALSE,FALSE,FALSE,FALSE,
<@$\linecontinoue$@>",",0,0,gmm,NA,NA,0,0,0,0,dyadic,FALSE
nonSymmetricObjective,real recip.,real reciprocated ties,realrecip,FALSE,,,gmm,FALSE,FALSE,FALSE,FALSE,FALSE,
<@$\linecontinoue$@>",",0,0,gmm,NA,NA,0,0,0,0,dyadic,FALSE
\end{lstlisting}
\end{listing*}

\item \sourcelinkFF{R}{sienaeffects.r}{includeEffect} flags the effect as included.

\item \sourcelinkFF{R}{initializeFRAN.r}{initializeFRAM} constructs a vector of EffectInfo objects vis \sourcelinkFF{src}{siena07setup.cpp}{effects}

\item Effects are instantiated \footnotesize{\sourcelinkF{src/model/effects}{EffectFactory.h}
    Note: The \sourcelinkF{src/model/effects}{AllEffects.h} attempts to list all effects.}
\begin{listing}
  \label{lst:factory}
  \caption{
    \sourcelinkF{src/model/effects}{EffectFacttory.h} consisting mainly of one big conditional.
  }
\begin{lstlisting}[language=c++]
Effect * EffectFactory::createEffect(const EffectInfo * pEffectInfo) const {
  /* huge if-else block with other effects... */
  if (effectName == "newrecip") {
    pEffect = new ReciprocityGMMEffect(pEffectInfo, ReciprocityGMMEffect::NEW);
  } else if (effectName == "realrecip") {
    pEffect = new ReciprocityGMMEffect(pEffectInfo, ReciprocityGMMEffect::REAL);
  } else if (effectName == "persistrecip") {
    pEffect = new ReciprocityGMMEffect(pEffectInfo, ReciprocityGMMEffect::PERSISTENT);
  }
  /* ...huge if-else block with other effects */
}
\end{lstlisting}
\end{listing}
\end{enumerate}

% \begin{description}
%   \item[edit \sourcelinkF{src/model/effects}{EffectFactory.cpp}:] Add an include to the top and a conditional to the \texttt{createEffect} function \ref{lst:factory}.
% \item[edit \sourcelinkF{data}{allEffects.csv}:] Add a line for each valid configuration.
% \end{description}

\section{TODO/Thoughts}

\paragraph{explain the contribution part a bit}

\paragraph{hint to other than network sides}

\paragraph{hint gmm functionality in the hierarchy}

\paragraph{difference between eval/endown/maintain implementation}

\paragraph{differences between functions (model slots)}

\paragraph{best practice for effects}

\begin{itemize}
  \item h / cpp
  \item thematic grouping
  \item cache usage
\end{itemize}

\paragraph{testing}

\begin{itemize}
  \item integration
\end{itemize}

\end{document}
