\documentclass[12pt,a4paper]{article}
\usepackage[pdftex,dvipsnames]{color}
\usepackage{graphicx,times}
\usepackage{amsmath}
\usepackage{amsfonts}
\usepackage{amssymb}
\usepackage{algorithm}
\usepackage{algorithmic}
\usepackage[pdfstartview={}]{hyperref}
\textheight=9.5in
\textwidth=6.0in
\topmargin=-0.5in
\evensidemargin=0in
\oddsidemargin=0in
\newcommand{\eps}{\varepsilon}
\newcommand{\abso}[1]{\;\mid#1\mid\;}
\renewcommand{\=}{\,=\,}
\newcommand{\+}{\,+\,}
% ----------------------------------------------------------------
\newcommand{\remark}[1]{\par\noindent{\color[named]{ProcessBlue}#1}\par}
\newcommand{\mcc}[2]{\multicolumn{#1}{c}{#2}}
\newcommand{\mcp}[2]{\multicolumn{#1}{c|}{#2}}
\newcommand{\nm}[1]{\textsf{\small #1}}
\newcommand{\nnm}[1]{\textsf{\small\textit{#1}}}
\newcommand{\nmm}[1]{\nnm{#1}}
\newcommand{\R}{{\sf R }}
\newcommand{\sfn}[1]{\textbf{\texttt{#1}}}
\newcommand{\Rn}{{\sf R}}
\newcommand{\rs}{{\sf RSiena}}
\newcommand{\RS}{{\sf RSiena }}
\newcommand{\SI}{{\sf Siena3 }}
\newcommand{\Sn}{{\sf Siena3}}
% no labels in list of references:
\makeatletter
\renewcommand\@biblabel{}
\makeatother

\hyphenation{Snij-ders Duijn DataSpecification dataspecification dependentvariable ModelSpecification}

% centered section headings with a period after the number;
% sans serif fonts for section and subsection headings
\renewcommand{\thesection}{\arabic{section}.}
\renewcommand{\thesubsection}{\thesection\arabic{subsection}}
\makeatletter
 \renewcommand{\section}{\@startsection{section}{1}
                {0pt}{\baselineskip}{0.5\baselineskip}
                {\centering\sffamily} }
 \renewcommand{\subsection}{\@startsection{subsection}{2}
                {0pt}{0.7\baselineskip}{0.3\baselineskip}
                {\sffamily} }
\makeatother

\newcommand{\ts}[1]{\par{\color[named]{Red}TS: #1}\par}

\renewcommand{\baselinestretch}{1.0} %% For line spacing.
\setlength{\parindent}{0pt}
\setlength{\parskip}{1ex plus1ex}
\begin{document}

\title{RSiena: processing of missing and structural values}
\author{Ruth Ripley}
\date{}
\maketitle

\centerline{\emph{\today}}
\bigskip
\section{Introduction}
This document describes the processing of missing and structurally fixed
values. It does not (yet!) include definitions of all the various missing data
attributes used in RSiena.

Nothing yet about ML.
\section{Outcome: see below for details}
\begin{description}
\item[networks]\hfill
\begin{enumerate}
\item Missing values are replaced by the most recent previous non-missing value,
  if any. Or zero.
\end{enumerate}
\item[behavior variables]\hfill
\begin{enumerate}
\item Missing values are replaced by the most recent previous non-missing value,
  if any, or the earliest later one, if any, or by the mode for the wave.
\item Centering is done by subtracting the mean each time the value is
  used in a relevant effect.
\item No structurals yet.
\end{enumerate}
\item[constant and changing covariates] \hfill
\begin{enumerate}
\item In siena07 and other functions relying on the C++ code (eg
  \verb|bayes|), missing values are replaced by 0, which is the mean of the
  centred data. Missing values are never altered in R.
\item Values are centered i.e. the ((global) mean is
  subtracted).
\end{enumerate}
\item[constant and changing dyadic covariates]\hfill
\begin{enumerate}
\item Missing values are replaced by the (global) mean when preparing the data
  for input to C++.
\item Values are centered by \emph{the relevant effect} subtracting the mean.
\end{enumerate}
\end{description}
\section{siena01 route}

\section{Object creation}

\subsection{networks: sienaNet}
\begin{enumerate}
\item For networks, all values must be missing or 0, 1, 10, 11
\end{enumerate}

\subsection{Data object}
\begin{description}
\item[networks]\hfill
\begin{enumerate}
\item Missing and structural values are excluded in calculation of distance
  between the waves.
\end{enumerate}
\item[behavior variables]\hfill
\begin{enumerate}
\item No mean is calculated.
\item \nnm{moreThan2} treats missing values as a separate category, unlike
  for covariates.
\item \nnm{poszvar} always true if there are missing values. Otherwise only if
    not all values are the same.
\item Missing values are excluded in calculation of distance between the waves.
\end{enumerate}
\item[constant and changing covariates]\hfill
\begin{enumerate}
\item Missing values are ignored in calculation of mean i.e. mean is average of
  non-missing values.
\item \nnm{moreThan2} ignores missing values.
\item \nnm{poszvar} always true if there are missing values. Otherwise only if
    not all values are the same.
\item Values are centered but NA's are never removed within R.
\item Range and similarity calculations ignore any missing values, excluding
  them from denominators as well.
\end{enumerate}
\item[dyadic covariates]\hfill
\begin{enumerate}
\item Missing values are ignored in calculation of mean.
\end{enumerate}
\end{description}

\section{Reports}

\section{Input to C++: initializeFRAN}
\begin{description}
\item[networks]\hfill
\begin{enumerate}
\item Missing values are replaced by the most recent previous value if
  any. Otherwise by 0. \emph{No carry back}.
\item Which values are treated as missing can be altered by composition change
  option. This may mean that a carried forward value is replaced
  by 0 later.
\item A new distance is calculated in R after these changes. Possibly is not
  used, but if it is used, its exact definition is not immediately apparent! See
  item below for details of composition change options. I think it is a hangover
  from days before the distance was calculated with the statistics.
\item Three edge lists are created, one for data which is to be used, one for
  missing indicators and one for indicators of structurally fixed values.
\end{enumerate}
\item[behavior variables] Missing values are replaced by a previous value or a
  later value or the mode for the wave. Done in \nnm{unpackBehavior}, called by
  \nnm{initializeFRAN}.
\item[constant and changing covariates] When the data is read in C++: it is then
  zero'd if missing, and the fact that it is missing is stored.
 (Uses R ISNAN macro in C++, a convenient flag.)
\item[dyadic covariates] When making edgelists in preparation for C++, in the
  unpack functions called by\nnm{initializeFRAN}, NA's are replaced by the mean.
\item[Composition change options]\hfill
\begin{description}
\item[1] Never treated as missing, except in the initial report.
Before the actor joins the network, row and column are set to 0
After the actor has joined, if later inactive, the previous
value is carried forward.
\item[2] Before the actor joins the network, row and column are set to 0 and the
  values are not treated as missing values. If temporarily missing after
  joining, the previous value is carried forward and this is not treated as a
  missing value. After finally leaving, any non-missing values in the data will
  be used, but they will be treated as missing values. If the data contained
  NA's then values are carried forward or zeroed as usual.
\item[3] When inactive the values are treated as missing. If any values are
  present they will be used. If NA's are present then values are carried
  forward or zeroed as usual.
\end{description}
\end{description}

\section{Processing in siena07}
\begin{description}
\item[networks] Each network has 3 network objects: one for the values and two
  boolean networks, one for missing indicators and one for structural
  indicators. In general, forward simulation does not concern itself with
  missing values until calculating statistics. Structural values are used
  alongside any composition change data when defining the flags indicating
  whether an actor is active or not.
\item[behavior variables] Centering is achieved when required by the relevant
  effect subtracting the overall mean. (Which is calculated after set up in C++,
  not R. Might be good idea to alter this as too easy to get reports and values
  used out of sync.)
\item[constant and changing covariates] The missing
  flags are used when calculating distance 2 covariate effects: if all the
  values which are being averaged are missing the result is treated as missing.
\item[dyadic covariates] The relevant effect subtracts the mean from the value
  each time the latter is requested.
\item[Statistics]\hfill
\begin{enumerate}
\item We create \emph{predictor} versions of the dependent variables which are
  used in any effect of which the dependent variable is not the owner. For
  networks these have all values missing at either end of the period set to
  zero. For behavior variables we only set to zero the values which are missing
  at the start of the period.
\item In the simulated values data for any actors who are inactive at the end of
  the period are set to the values at the start of the period so as not to
  affect the statistics. This is done in a routine named \nnm{setLeaverBack} but
  in fact it affects all actors inactive at the time, not just leavers during
  the period.
\item Then each dependent variable is processed in turn. The three types of
  effects are processed differently, and differently for networks and behavior
  variables:
  \begin{description}
  \item[networks]\hfill
    \begin{description}
    \item[evaluation]\hfill
      \begin{enumerate}
      \item In a copy of the current state of the owner dependent
        variable we make the following changes:
        \begin{description}
        \item[missing] Values missing at either end of the period are set to 0.
        \item[structural] There are two cases to consider: initial calculation
          of targets, and calculation of simulated statistics after
          simulation. In the former case we have used the data for the end of
          the period as the current state so we copy values structurally fixed
          at the start of the period from the data to the state. In the other
          case we copy values structurally fixed at the end to our state. (The
          same two changes are made in each case, but only one has any effect.)
        \end{description}
      \item Since a network is not used at the same time as a predictor and an
        owner we temporarily overwrite the predictor state's copy of this
        network with our changed copy so we can use the predictor state for all
        our calculations.
      \end{enumerate}
    \item[endowment] Here we create a \nnm{LostTieNetwork} and also overwrite
      the predictor network.
      \begin{description}
      \item[lostTieNetwork]\hfill
        \begin{enumerate}
        \item A copy is made of the initial state of the network, from the data
        \item In a copy of the current state of the owner dependent variable,
          any values that are structurally fixed are changed as for evaluation
          effects.
        \item Any values which are 1 in our changed copy of the current state
          are set to zero in the lost tie network.
        \item Any values missing at the end of the period are set to zero in the
          lost tie network. ?? why not start too?
        \end{enumerate}
      \item[predictor network]
        \begin{enumerate}
        \item Values missing at the start of the period are set to zero in a
          copy of the current state of the network. (Values missing at the end
          are not changed).
        \item Since a network is not used at the same time as a predictor and an
          owner we temporarily overwrite the predictor state's copy of this
          network with our changed copy so we can use the predictor state for
          all our calculations.
        \end{enumerate}
      \end{description}
    \item[creation] Here we create a \nnm{GainedTieNetwork} and also overwrite
      the predictor network as for evaluation effects.
      \begin{description}
      \item[predictor network] A changed copy of the current state is created as
        for evaluation effects.
      \item[GainedTieNetwork]
         A copy is made of our altered predictor network, and any values
          missing or 1 at the start of the period are set to zero.
      \end{description}
    \end{description}
  \item[behavior]
    \begin{enumerate}
    \item A copy of the current state of values (less overall mean) is made and
      any values missing at start or end of the period are set to zero.
    \item A vector of differences, start of period value less (unaltered)
      current state is created. Entries missing at either end of the period are
      set to zero.
    \item Evaluation effects are calculated using the altered current centred
      values.
    \item Endowment and creation effects are calculated using the altered
      current centred values and the differences.
    \end{enumerate}
  \end{description}
\end{enumerate}
\end{description}
\end{document}
